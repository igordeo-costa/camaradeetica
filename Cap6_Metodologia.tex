\chapter{Metodologia}
Este trabalho pretende valer-se exclusivamente de metodologia experimental, visto vincular-se ao modelo de pesquisa rotineiramente desenvolvido no LAPAL-PUC-Rio e ser o método que permite um controle preciso e rigoroso dos inúmeros fatores que governam o fenômeno em questão. Já que fatores semânticos e pragmáticos podem interferir e até mesmo determinar certas leituras, investigar o comportamento do \emph{processador humano} depende de um refinado controle das potenciais inteferências, linguísticas ou não, que possam estar presentes no caso em questão. A metodologia experimental é aquela que melhor se adéqua, portanto, a esse tipo de estudo.

Ademais, como informado na revisão da literatura, quando se trata de técnica \emph{off-line}, o escopo de quantificadores parece convergir, na grande maioria dos casos, em uma leitura preferencial das sentenças estudadas. Entretanto, quando se trata de técnica \emph{on-line}, os resultados são conflitantes. Sendo assim, o uso de ambas as metodologias mostram-se fundamentais para o estudo do comportamento de \emph{todo+NP} no português brasileiro.

Sendo assim, o primeiro experimento que pretendemos realizar será uma tarefa de leitura automonitorada (\emph{self-paced reading}) com julgamento de gramaticalidade usando a \emph{escala Likert} (REFERÊNCIAS). Este estudo teria caráter normativo, visando estabelecer um conjunto mínimo de sentenças que demonstrassem \textcolor{red}{``ativar''} comportamentos semelhantes nos falantes e fossem efetivamente ambíguas quanto às leituras distributiva e coletiva. A partir da discussão sobre as propriedades linguísticas de \emph{todo} no português brasileiro, pode-se dizer que deveremos elaborar sentenças semelhantes àquelas abaixo:

\begin{enumerate}
  \item O vendedor mostrou toda pulseira de ouro para um comprador.
  \item O vendedor mostrou uma pulseira para todo comprador de joias.
\end{enumerate}

Essa tarefa deverá ser realizada exclusivamente em ambiente virtual, com o uso da ferramenta de elaboração de experimentos disponível \emph{online} chamada \emph{Ibex Farm}, acessível em: \emph{https://spellout.net/ibexfarm/}. O comportamento dos falantes ao julgar tais sentenças permitirá a elaboração de um conjunto de frases normatizadas que se comportam de modo semlhante, garantindo maior rigor na construção do segundo experimento.

Nessa segunda atividade, pretendemos realizar uma tarefa de leitura automonitorada seguida de uma frase continuativa com um nome anafórico que retome o substantivo encabeçado pelo quantificador indefinido \emph{um}. Tais estímulos deverão se assemelhar àqueles abaixo:

\begin{enumerate}[resume]
  \item O vendedor mostrou toda pulseira de ouro para um comprador,\\ mas \textbf{o homem/os homens} não era/eram muito decidido/decididos.
  \item O vendedor mostrou uma pulseira para todo comprador de joias,\\ mas \textbf{a peça/as peças} não era/eram muito bonita/bonitas.
\end{enumerate}

Essa atividade permitirá a medição do tempo de reação do participante nas posições do nome anafórico, do verbo que se segue e do advérbio seguinte, de modo que será possível verificar a reação imediata, espontânea e não mediada por integração de informação contextual e/ou semantico-pragmática, permitindo visualizar a leitura que o \emph{parser} realiza da sentença. Assim, será possível visumbrar, também, se há ou não uma leitura preferencial ou se a especificação fica vaga.

Essa tarefa deverá ser realizada em ambiente virtual, usando, também, o \emph{Ibex Farm}. No entanto, como ela envolve a medição do tempo de reação dos participantes, seria interessante que, caso as circunstâncias permitam, ela seja complementada com uma tarefa presencial, no laboratório, onde não há potenciais distrações externas e o experimento está mais prontamente sob controle do pesquisador.

Em paralelo a tais experimentos, pretendemos, também, realizar um experimento de leitura automonitorada com \emph{twist}, técnica desenvolvida por Patson \& Warren (2010)~\cite{PatWarren2010} e usada por Dwivedi \& Gibson (2017)~\cite{DwiGib2017}. Apesar de a técnica do experimento apresentado anteriormente mensurar o comportamento do \emph{parser} em tempo real, ela o faz na sentença continuativa, não na sentença quantificada. Essa técnica, no entanto, permite investigar se o escopo é estabelecido \emph{in loco}, ou seja, na própria sentença quantificada, assim que o participante se depara com o segundo quantificador da sentença. As frases utilizadas serão as mesmas normatizadas pelo experimento \emph{off-line} já discutido.

Como as tarefas anteriores, essa também se valerá do \emph{Ibex Farm}, e, já que envolve controle de tempo, seria ideal que pudesse ser realizada no laboratório, onde as condições são mais controladas.

A depender, também, das circunstâncias sanitárias do país, pretendemos realizar uma tarefa de rastreamento ocular (\emph{eye tracking}), essa exclusivamente em laboratório, que possui recursos tecnológicos próprios para tal, vinculada ao \emph{paradigma do mundo visual} (REFERÊNCIAS). Com esse tipo de técnica seria possível mensurar a movimentação do olhar dos participantes e perceber a leitura que fazem das sentenças em tempo real, antes mesmo que ele possa realizar qualquer reflexão metalinguística sobre o fenômeno. A construção das sentenças seguiria os mesmos padrões das tarefas anteriores e dependeria, obviamente, dos resultados obtidos naqueles experimentos.

Por fim, a depender dos resultados obtidos e das circunstâncias, pretendemos realizar um experimento de produção, de caráter exploratório, em que o participante, após ler/ouvir a sentença quantificada, deveria dar uma continuação a ela a partir de alguma pista. Essa tarefa é importante porque permite verificar a interpretação de fato construída pelo falante e não a sua reação a interpretações já dadas, caso de todos os experimmentos anteriores.

Em todas as tarefas, os riscos de participação são mínimos. As sentenças serão construídas de modo a evitar quaisquer expressões ou descrever quaisquer cenas que possam por ventura causar constrangimentos aos participantes. Entretanto, ainda assim é possível que o participante sinta leve desconforto por se manter sentado(a) e parcialmente imóvel durante a realização das tarefas. Esse desconforto será tentativamente minimizado ao construirmos experimentos com poucos estímulos experimentais (aproximadamente 50 sentenças) e, quando feitos no laboratório, disponibilizaremos um local tranquilo e confortável com um pequeno intervalo na metade da atividade se o participante desejar.

O participante também será informado que se encontra livre para encerrar a atividade a qualquer momento sem qualquer prejuízo ou constrangimento à sua pessoa.

Especificamente sobre o uso de rastreador ocular (\emph{eye tracker}, os riscos são mínimos, similares aos de atividades diárias como uso de computador e de televisão. A luz infravermelha invisível emitida pelo aparelho assemelha-se a luzes naturais e artificiais (como o sol, o fogo, velas e certas lâmpadas) presentes em vários ambientes. O aparelho é testado de acordo com as normas europeias de segurança, de forma a ser considerado inofensivo aos seres humanos.

Por fim, garantiremos ao participante -- e ele será informado a respeito antes do início da atividade -- que suas informações serão tratadas com o mais absoluto sigilo e confidencialidade, de modo a preservar a sua identidade. Os resultados da pesquisa serão divulgados em eventos e publicações científicas, para fins acadêmicos apenas, sendo mantido o anonimato dos participantes.
