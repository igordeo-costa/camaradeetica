\chapter{Anexos}
Apresentamos, abaixo, o textos dos Termos de Consentimento Livre e Esclarecido (TCLE) a serem fornecidos aos participantes antes da realização da tarefa a fim de garantir que estejam bem informados quanto à proposta da pesquisa e seus riscos potenciais. Todos os termos impressos serão encabeçados pela logomarca da PUC-Rio, pela expressão ``Termo de Consentimento Livre e Esclarecido'' e pelo nome da tarefa a ser realizada (\emph{v.g.} ``Atividade de leitura com captura de dados oculares'').

\section{Atividade de leitura e julgamento \emph{Online}}
O termo de consentimento abaixo será usado para a tarefa de julgamento de gramaticalidade com \emph{escala Likert}.

\begin{center}
  \textbf{Termo de Consentimento Livre e Esclarecido}
\end{center}
Convidamos você a participar como voluntário(a) da pesquisa intitulada ``Estruturas distributivas no Português Brasileiro: aspectos linguísticos e processuais'', de responsabilidade do doutorando Igor de Oliveira Costa, orientando da profª. Erica dos Santos Rodrigues, do Programa de Pós-Graduação Estudos da Linguagem (PPGEL), PUC-Rio.
\\
\\
OBJETIVOS: O objetivo da pesquisa é investigar como os falantes do Português Brasileiro interpretam expressões linguísticas usadas para especificar a quantos elementos a frase está remetendo.
\\
\\
JUSTIFICATIVA: O estudo irá contribuir (i) para o desenvolvimento de pesquisas em Linguística Teórica sobre as estruturas de quantificação no Português Brasileiro e (ii) para a avaliação, em Psicolinguística, de como as pessoas compreendem as referidas estruturas em situações controladas.
\\
\\
PROCEDIMENTOS: Nesta tarefa você irá ler um conjunto de sentenças em português e informará a sua interpretação sobre elas. Não se trata de um teste de gramática, com respostas certas ou erradas. A proposta é registrar sua resposta mais natural e espontânea sobre as estruturas da língua. A realização da tarefa dura alguns minutos apenas.
\\
\\
Antes de realizar a tarefa, você preencherá uma ficha com alguns dados pessoais, como idade, escolaridade e gênero/sexo. Seu nome não será registrado e em hipótese alguma faremos referência à sua identidade.
\\
\\
RISCOS: É possível que você sinta leve desconforto e cansaço por se manter sentado(a) e concentrado na leitura das frases apresentadas. Entretanto, a atividade é de curta duração, sendo mínimos os riscos envolvidos na realização da tarefa, similares aos de atividades diárias com o uso de computador.
\\
\\
BENEFÍCIOS: Você não pagará e nem será remunerado(a) por sua participação. Sua participação voluntária irá, contudo, contribuir para as pesquisas, tanto em Linguística Teórica quanto em Psicolinguística, sobre estruturas sintáticas envolvendo relações de quantificação.
\\
\\
DIVULGAÇÃO E CONFIDENCIALIDADE: Esclarecemos que sua participação é totalmente voluntária, podendo: recusar-se a participar, ou mesmo desistir a qualquer momento, sem que isto acarrete qualquer ônus ou prejuízo a sua pessoa. Esclarecemos, também, que suas informações serão tratadas com o mais absoluto sigilo e confidencialidade, de modo a preservar a sua identidade. Os resultados da pesquisa serão divulgados em eventos e publicações científicas, sendo mantido o anonimato dos participantes.
\\
\\
INFORMAÇÕES ADICIONAIS: Contatos para esclarecimentos de dúvidas sobre a pesquisa e seus aspectos éticos:
\\
\\
Câmara de Ética em Pesquisa da PUC-Rio (CEPq-PUC-Rio), situado  na Rua Marquês de São Vicente, 225 -- Edifício Kenedy, 2º andar. Gávea -- Rio de Janeiro/RJ, CEP: 22453-900; Telefone: + 55 (21) 3527-1618.
\\
\\
Programa de Pós-Graduação Estudos da Linguagem -- Departamento de Letras da PUC--Rio, situado na Rua Marquês de São Vicente, 225 -- Edifício Pe. Leonel Franca, 3º andar. Gávea -- Rio de Janeiro/RJ, CEP: 22451-900; Telefone: +55 (21) 3527-1297
\\
\\
Caso deseje esclarecimentos adicionais sobre a pesquisa, os pesquisadores podem ser contactados nos seguintes e-mails:
\\
\\
E-mail do doutorando: igordeo.costa@gmail.com \\
E-mail da orientadora: ericasr@puc-rio.br
\\
\\
Este termo de consentimento será arquivado pelos pesquisadores responsáveis e uma cópia será enviada a você por e-mail. Os dados coletados na pesquisa ficarão arquivados com o pesquisador responsável por um período de 5 (cinco) anos. Decorrido este tempo, os pesquisadores avaliarão os documentos para a sua destinação final, de acordo com a legislação vigente. Os pesquisadores tratarão a sua identidade com padrões profissionais de sigilo, atendendo a legislação brasileira (Resoluções Nº 510/16 e Nº 466/12 do Conselho Nacional de Saúde), utilizando as informações somente para os fins acadêmicos e científicos.
\\
\\
Declaro que li com atenção o termo de consentimento livre e esclarecido da pesquisa “Estruturas distributivas no Português Brasileiro: aspectos linguísticos e processuais” e acredito estar suficientemente informado sobre o estudo, ficando claro que minha participação é voluntária e que posso retirar este consentimento a qualquer momento sem penalidades.
\\
\\
Ao clicar nessa opção, expresso minha concordância em participar deste estudo.
\\
\\
ACEITO PARTICIPAR

\section{Atividades de leitura automonitorada \emph{Online} e presencial}
O termo que se segue será usado para os experimentos de leitura automonitorada e leitura automonitorada com \emph{twist}.

\begin{center}
  \textbf{Termo de Consentimento Livre e Esclarecido}
\end{center}
Convidamos você a participar como voluntário(a) da pesquisa intitulada ``Estruturas distributivas no Português Brasileiro: aspectos linguísticos e processuais'', de responsabilidade do doutorando Igor de Oliveira Costa, orientando da profª. Erica dos Santos Rodrigues, do Programa de Pós-Graduação Estudos da Linguagem (PPGEL), PUC-Rio.
\\
\\
OBJETIVOS: O objetivo da pesquisa é investigar como os falantes do Português Brasileiro interpretam expressões linguísticas usadas para especificar a quantos elementos a frase está remetendo.
\\
\\
JUSTIFICATIVA: O estudo irá contribuir (i) para o desenvolvimento de pesquisas em Linguística Teórica sobre as estruturas de quantificação no Português Brasileiro e (ii) para a avaliação, em Psicolinguística, de como as pessoas compreendem as referidas estruturas em situações controladas.
\\
\\
PROCEDIMENTOS: Nesta tarefa você irá ler um conjunto de sentenças em português e \textcolor{blue}{responderá, apertando teclas do computador, algumas perguntas de compreensão sobre elas}. Não se trata de um teste de gramática, com respostas certas ou erradas. A proposta é registrar sua resposta mais natural e espontânea sobre as estruturas da língua. A realização da tarefa dura alguns minutos apenas. \textcolor{blue}{Durante a realização da tarefa, o computador registrará o tempo que você leva para passar de uma palavra a outra.}
\\
\\
Antes de realizar a tarefa, você preencherá uma ficha com alguns dados pessoais, como idade, escolaridade e gênero/sexo. Seu nome não será registrado e em hipótese alguma faremos referência à sua identidade.
\\
\\
RISCOS: É possível que você sinta leve desconforto e cansaço por se manter sentado(a) e concentrado na leitura das frases apresentadas. Entretanto, a atividade é de curta duração, sendo mínimos os riscos envolvidos na realização da tarefa, similares aos de atividades diárias com o uso de computador.
\\
\\
BENEFÍCIOS: Você não pagará e nem será remunerado(a) por sua participação. Sua participação voluntária irá, contudo, contribuir para as pesquisas, tanto em Linguística Teórica quanto em Psicolinguística, sobre estruturas sintáticas envolvendo relações de quantificação.
\\
\\
DIVULGAÇÃO E CONFIDENCIALIDADE: Esclarecemos que sua participação é totalmente voluntária, podendo: recusar-se a participar, ou mesmo desistir a qualquer momento, sem que isto acarrete qualquer ônus ou prejuízo a sua pessoa. Esclarecemos, também, que suas informações serão tratadas com o mais absoluto sigilo e confidencialidade, de modo a preservar a sua identidade. Os resultados da pesquisa serão divulgados em eventos e publicações científicas, sendo mantido o anonimato dos participantes.
\\
\\
INFORMAÇÕES ADICIONAIS: Contatos para esclarecimentos de dúvidas sobre a pesquisa e seus aspectos éticos:
\\
\\
Câmara de Ética em Pesquisa da PUC-Rio (CEPq-PUC-Rio), situado  na Rua Marquês de São Vicente, 225 -- Edifício Kenedy, 2º andar. Gávea -- Rio de Janeiro/RJ, CEP: 22453-900; Telefone: + 55 (21) 3527-1618.
\\
\\
Programa de Pós-Graduação Estudos da Linguagem -- Departamento de Letras da PUC--Rio, situado na Rua Marquês de São Vicente, 225 -- Edifício Pe. Leonel Franca, 3º andar. Gávea -- Rio de Janeiro/RJ, CEP: 22451-900; Telefone: +55 (21) 3527-1297
\\
\\
Caso deseje esclarecimentos adicionais sobre a pesquisa, os pesquisadores podem ser contactados nos seguintes e-mails:
\\
\\
E-mail do doutorando: igordeo.costa@gmail.com \\
E-mail da orientadora: ericasr@puc-rio.br
\\
\\
Este termo de consentimento será arquivado pelos pesquisadores responsáveis e uma cópia será enviada a você por e-mail. Os dados coletados na pesquisa ficarão arquivados com o pesquisador responsável por um período de 5 (cinco) anos. Decorrido este tempo, os pesquisadores avaliarão os documentos para a sua destinação final, de acordo com a legislação vigente. Os pesquisadores tratarão a sua identidade com padrões profissionais de sigilo, atendendo a legislação brasileira (Resoluções Nº 510/16 e Nº 466/12 do Conselho Nacional de Saúde), utilizando as informações somente para os fins acadêmicos e científicos.
\\
\\
\noindent \textbf{Para o caso de atividade \emph{online}, será adicionado o seguinte trecho:}\\
Declaro que li com atenção o termo de consentimento livre e esclarecido da pesquisa ``Estruturas distributivas no Português Brasileiro: aspectos linguísticos e processuais'' e acredito estar suficientemente informado sobre o estudo, ficando claro que minha participação é voluntária e que posso retirar este consentimento a qualquer momento sem penalidades.
\\
\\
Ao clicar nessa opção, expresso minha concordância em participar deste estudo.
\\
\\
ACEITO PARTICIPAR
\\
\noindent \textbf{Para o caso de atividade presencial, será adicionado o seguinte trecho:}\\
\\
\noindent CONSENTIMENTO\\
Eu, \rule[-1mm]{8cm}{0.3mm}, CPF \rule[-1mm]{5cm}{0.3mm}, após a leitura deste documento e de ter tido a oportunidade de esclarecer todas as minhas dúvidas, acredito estar suficientemente informado(a), ficando claro que minha participação no projeto ``Estruturas distributivas no Português Brasileiro: aspectos linguísticos e processuais'' é voluntária e que posso retirar este consentimento a qualquer momento sem penalidades ou perda de qualquer benefício. Estou ciente também dos objetivos da pesquisa, dos procedimentos aos quais serei submetido(a), dos possíveis danos ou riscos deles provenientes e da garantia de confidencialidade e esclarecimentos sempre que desejar.
Diante do exposto expresso minha concordância de espontânea vontade em participar deste estudo.

\begin{center}
  Rio de Janeiro, \rule[-1mm]{1cm}{0.3mm} de \rule[-1mm]{3cm}{0.3mm} de \rule[-1mm]{1.5cm}{0.3mm}.\\
  \vspace{1cm}
  \rule[-1mm]{8cm}{0.3mm}\\
  (Assinatura do participante da pesquisa)\\
  \vspace{1cm}
  \rule[-1mm]{8cm}{0.3mm}\\
  (Igor de Oliveira Costa -- pesquisador responsável)
\end{center}

\section{Atividade de rastreamento ocular}
O termo de consentimento abaixo será usado para a tarefa de rastreamento ocular a ser realizada presencialmente nas instalações físicas do LAPAL/PUC-Rio ou em outras dependências da PUC-Rio e/ou de outras universidades.

\begin{center}
  \textbf{Termo de Consentimento Livre e Esclarecido}
\end{center}
Você está sendo convidado(a) a participar da Pesquisa ``Estruturas distributivas no Português Brasileiro: aspectos linguísticos e processuais'', de responsabilidade do doutorando Igor de Oliveira Costa, orientando da profa. Erica dos Santos Rodrigues, do Programa de Pós-Graduação Estudos da Linguagem (PPGEL), PUC-Rio. Antes de aceitar participar desta pesquisa, é necessária sua compreensão a respeito das informações e das instruções contidas neste documento. Os pesquisadores responderão todas as dúvidas antes que decida participar. A qualquer momento, poderá interromper sua participação na pesquisa sem sofrer qualquer penalização ou constrangimento.
\\
\\
OBJETIVOS: O objetivo da pesquisa é investigar como os falantes do Português Brasileiro interpretam expressões linguísticas usadas para especificar a quantos elementos a frase está remetendo. Na atividade experimental de que você irá participar serão capturados e registrados seus movimentos oculares durante a leitura de sentenças e/ou apresentação de imagens acompanhadas da escuta dessas frases, de modo a avaliar hipóteses sobre como as estruturas sintáticas testadas são compreendidas.
\\
\\
JUSTIFICATIVA: O estudo irá contribuir (i) para o desenvolvimento de pesquisas em Linguística Teórica sobre as estruturas de quantificação no Português Brasileiro e (ii) para a avaliação, em Psicolinguística, de como as pessoas compreendem as referidas estruturas em situações controladas.
\\
\\
PROCEDIMENTOS: Você irá ler ou ouvir um conjunto de sentenças em português, que podem estar acompanhadas ou não de imagens, com apresentação em computador, e irá responder a algumas perguntas simples de compreensão. Não se trata de um teste de gramática, com respostas certas ou erradas. A proposta é registrar sua resposta mais natural e espontânea sobre as estruturas da língua. Durante a realização da tarefa, seus movimentos oculares serão gravados com uso de um equipamento de rastreamento ocular. O equipamento detectará os movimentos oculares a partir de reflexos gerados na córnea por uma luz infravermelha emitida pelo equipamento, similar a luzes naturais e artificiais. Será necessário realizar, antes da apresentação dos estímulos, um procedimento de ``calibração'' do equipamento, durante o qual você será solicitado(a) a olhar para pontos apresentados na tela de computador. Você receberá orientação para se posicionar de maneira confortável na cadeira em que se encontra e será solicitado que evite realizar movimentos bruscos durante a sessão para preservar a qualidade da calibração realizada. A sessão experimental terá duração aproximada de 20 minutos. Ao final da tarefa, você será solicitado(a) a preencher um questionário sociodemográfico, com algumas perguntas sobre idade, sexo, escolaridade e história linguística.
\\
\\
DESCONFORTOS E RISCOS ESPERADOS: É possível que você sinta leve desconforto por se manter sentado(a) e parcialmente imóvel durante a sessão. No entanto, buscamos minimizar ao máximo esse desconforto, realizando a tarefa em local tranquilo e confortável e fazendo um pequeno intervalo na metade da atividade se desejar. Você encontra-se livre para encerrar a atividade a qualquer momento. Os riscos envolvidos na realização da tarefa são mínimos, similares aos de atividades diárias como uso de computador e de televisão. A luz infravermelha invisível emitida pelo aparelho assemelha-se a luzes naturais e artificiais (como o sol, o fogo, velas e certas lâmpadas) presentes em vários ambientes. O aparelho é testado de acordo com as normas europeias de segurança, de forma a ser considerado inofensivo aos seres humanos.
\\
\\
BENEFÍCIOS PARA OS PARTICIPANTES: Você não pagará e nem será remunerado(a) por sua participação. Sua participação voluntária irá, contudo, contribuir para as pesquisas, tanto em Linguística Teórica quanto em Psicolinguística, sobre estruturas sintáticas envolvendo relações de quantificação.
\\
\\
DIVULGAÇÃO E CONFIDENCIALIDADE: Esclarecemos que sua participação é totalmente voluntária, podendo: recusar-se a participar, ou mesmo desistir a qualquer momento, sem que isto acarrete qualquer ônus ou prejuízo a sua pessoa. Esclarecemos, também, que suas informações serão tratadas com o mais absoluto sigilo e confidencialidade, de modo a preservar a sua identidade. Os resultados da pesquisa serão divulgados em eventos e publicações científicas, sendo mantido o anonimato dos participantes.
\\
\\
INFORMAÇÕES ADICIONAIS: Contatos para esclarecimentos de dúvidas sobre a pesquisa e seus aspectos éticos:
\\
\\
Câmara de Ética em Pesquisa da PUC-Rio (CEPq-PUC-Rio), situado  na Rua Marquês de São Vicente, 225 -- Edifício Kenedy, 2º andar. Gávea -- Rio de Janeiro/RJ, CEP: 22453-900; Telefone: + 55 (21) 3527-1618.
\\
\\
Programa de Pós-Graduação Estudos da Linguagem -- Departamento de Letras da PUC--Rio, situado na Rua Marquês de São Vicente, 225 -- Edifício Pe. Leonel Franca, 3º andar. Gávea -- Rio de Janeiro/RJ, CEP: 22451-900; Telefone: +55 (21) 3527-1297
\\
\\
Caso deseje esclarecimentos adicionais sobre a pesquisa, os pesquisadores podem ser contactados nos seguintes e-mails:
\\
\\
E-mail do doutorando: igordeo.costa@gmail.com \\
E-mail da orientadora: ericasr@puc-rio.br
\\
\\
Este termo de consentimento será arquivado pelos pesquisadores responsáveis e uma cópia será enviada a você por e-mail. Os dados coletados na pesquisa ficarão arquivados com o pesquisador responsável por um período de 5 (cinco) anos. Decorrido este tempo, os pesquisadores avaliarão os documentos para a sua destinação final, de acordo com a legislação vigente. Os pesquisadores tratarão a sua identidade com padrões profissionais de sigilo, atendendo a legislação brasileira (Resoluções Nº 510/16 e Nº 466/12 do Conselho Nacional de Saúde), utilizando as informações somente para os fins acadêmicos e científicos.
\\
\\
\noindent CONSENTIMENTO \\
Eu, \rule[-1mm]{8cm}{0.3mm}, CPF \rule[-1mm]{5cm}{0.3mm}, após a leitura deste documento e de ter tido a oportunidade de esclarecer todas as minhas dúvidas, acredito estar suficientemente informado(a), ficando claro que minha participação no projeto ``Estruturas distributivas no Português Brasileiro: aspectos linguísticos e processuais'' é voluntária e que posso retirar este consentimento a qualquer momento sem penalidades ou perda de qualquer benefício. Estou ciente também dos objetivos da pesquisa, dos procedimentos aos quais serei submetido(a), dos possíveis danos ou riscos deles provenientes e da garantia de confidencialidade e esclarecimentos sempre que desejar.
Diante do exposto expresso minha concordância de espontânea vontade em participar deste estudo.

\begin{center}
  Rio de Janeiro, \rule[-1mm]{1cm}{0.3mm} de \rule[-1mm]{3cm}{0.3mm} de \rule[-1mm]{1.5cm}{0.3mm}.\\
  \vspace{1cm}
  \rule[-1mm]{8cm}{0.3mm}\\
  (Assinatura do participante da pesquisa)\\
  \vspace{1cm}
  \rule[-1mm]{8cm}{0.3mm}\\
  (Igor de Oliveira Costa -- pesquisador responsável)
\end{center}
\vspace{2cm}

\section{Atividade de produção eliciada}

\begin{center}
  \textbf{Termo de Consentimento Livre e Esclarecido}
\end{center}
Convidamos você a participar como voluntário(a) da pesquisa intitulada ``Estruturas distributivas no Português Brasileiro: aspectos linguísticos e processuais'', de responsabilidade do doutorando Igor de Oliveira Costa, orientando da profª. Erica dos Santos Rodrigues, do Programa de Pós-Graduação Estudos da Linguagem (PPGEL), PUC-Rio.
\\
\\
OBJETIVOS: O objetivo da pesquisa é investigar como os falantes do Português Brasileiro interpretam expressões linguísticas usadas para especificar a quantos elementos a frase está remetendo.
\\
\\
JUSTIFICATIVA: O estudo irá contribuir (i) para o desenvolvimento de pesquisas em Linguística Teórica sobre as estruturas de quantificação no Português Brasileiro e (ii) para a avaliação, em Psicolinguística, de como as pessoas compreendem as referidas estruturas em situações controladas.
\\
\\
PROCEDIMENTOS: Nesta tarefa você irá ler um conjunto de sentenças em português e \textcolor{blue}{produzirá, a partir de pistas visuais fonecidas na tela do computador, frases que permitam dar continuidade a tais sentenças}. Não se trata de um teste de gramática, com respostas certas ou erradas. A proposta é registrar sua resposta mais natural e espontânea sobre as estruturas da língua. A realização da tarefa dura alguns minutos apenas.
\\
\\
Antes de realizar a tarefa, você preencherá uma ficha com alguns dados pessoais, como idade, escolaridade e gênero/sexo. Seu nome não será registrado e em hipótese alguma faremos referência à sua identidade.
\\
\\
RISCOS: É possível que você sinta leve desconforto e cansaço por se manter sentado(a) e concentrado na leitura das frases apresentadas. Entretanto, a atividade é de curta duração, sendo mínimos os riscos envolvidos na realização da tarefa, similares aos de atividades diárias com o uso de computador.
\\
\\
BENEFÍCIOS: Você não pagará e nem será remunerado(a) por sua participação. Sua participação voluntária irá, contudo, contribuir para as pesquisas, tanto em Linguística Teórica quanto em Psicolinguística, sobre estruturas sintáticas envolvendo relações de quantificação.
\\
\\
DIVULGAÇÃO E CONFIDENCIALIDADE: Esclarecemos que sua participação é totalmente voluntária, podendo: recusar-se a participar, ou mesmo desistir a qualquer momento, sem que isto acarrete qualquer ônus ou prejuízo a sua pessoa. Esclarecemos, também, que suas informações serão tratadas com o mais absoluto sigilo e confidencialidade, de modo a preservar a sua identidade. Os resultados da pesquisa serão divulgados em eventos e publicações científicas, sendo mantido o anonimato dos participantes.
\\
\\
INFORMAÇÕES ADICIONAIS: Contatos para esclarecimentos de dúvidas sobre a pesquisa e seus aspectos éticos:
\\
\\
Câmara de Ética em Pesquisa da PUC-Rio (CEPq-PUC-Rio), situado  na Rua Marquês de São Vicente, 225 -- Edifício Kenedy, 2º andar. Gávea -- Rio de Janeiro/RJ, CEP: 22453-900; Telefone: + 55 (21) 3527-1618.
\\
\\
Programa de Pós-Graduação Estudos da Linguagem -- Departamento de Letras da PUC--Rio, situado na Rua Marquês de São Vicente, 225 -- Edifício Pe. Leonel Franca, 3º andar. Gávea -- Rio de Janeiro/RJ, CEP: 22451-900; Telefone: +55 (21) 3527-1297
\\
\\
Caso deseje esclarecimentos adicionais sobre a pesquisa, os pesquisadores podem ser contactados nos seguintes e-mails:
\\
\\
E-mail do doutorando: igordeo.costa@gmail.com \\
E-mail da orientadora: ericasr@puc-rio.br
\\
\\
Este termo de consentimento será arquivado pelos pesquisadores responsáveis e uma cópia será enviada a você por e-mail. Os dados coletados na pesquisa ficarão arquivados com o pesquisador responsável por um período de 5 (cinco) anos. Decorrido este tempo, os pesquisadores avaliarão os documentos para a sua destinação final, de acordo com a legislação vigente. Os pesquisadores tratarão a sua identidade com padrões profissionais de sigilo, atendendo a legislação brasileira (Resoluções Nº 510/16 e Nº 466/12 do Conselho Nacional de Saúde), utilizando as informações somente para os fins acadêmicos e científicos.
\\
\\
\noindent CONSENTIMENTO \\
Eu, \rule[-1mm]{8cm}{0.3mm}, CPF \rule[-1mm]{5cm}{0.3mm}, após a leitura deste documento e de ter tido a oportunidade de esclarecer todas as minhas dúvidas, acredito estar suficientemente informado(a), ficando claro que minha participação no projeto ``Estruturas distributivas no Português Brasileiro: aspectos linguísticos e processuais'' é voluntária e que posso retirar este consentimento a qualquer momento sem penalidades ou perda de qualquer benefício. Estou ciente também dos objetivos da pesquisa, dos procedimentos aos quais serei submetido(a), dos possíveis danos ou riscos deles provenientes e da garantia de confidencialidade e esclarecimentos sempre que desejar.
Diante do exposto expresso minha concordância de espontânea vontade em participar deste estudo.

\begin{center}
  Rio de Janeiro, \rule[-1mm]{1cm}{0.3mm} de \rule[-1mm]{3cm}{0.3mm} de \rule[-1mm]{1.5cm}{0.3mm}.\\
  \vspace{1cm}
  \rule[-1mm]{8cm}{0.3mm}\\
  (Assinatura do participante da pesquisa)\\
  \vspace{1cm}
  \rule[-1mm]{8cm}{0.3mm}\\
  (Igor de Oliveira Costa -- pesquisador responsável)
\end{center}
