\chapter{Resultados de experimentos em inglês}

\textcolor{red}{Teste de bibliografia:\\
Kurtzman \& MacDonald (1993)~\cite{KMac1993}~\cite{KMac1993}\\
Tunstall (1998)~\cite{Tunstall1998}~\cite{Tunstall1998}\\
Anderson (2004)~\cite{Anderson2004}~\cite{Anderson2004}\\
Paterson et al. (2007)~\cite{paterson2007}\\}

Quando se trata da verificação das relações de escopo, a literatura em língua inglesa apresenta uma série de trabalhos que investiga o comportamento do processador humano em sentenças duplamente quantificadas. Os resultados, no entanto, são muitas vezes conflitantes, mesmo quando se olha apenas para a relação entre o quantificador universal \emph{every} -- que, semanticamente, é o elemento que mais se aproxima do \emph{todo} em português -- e o indefinido \emph{a}. Nesta seção, faremos uma revisão de alguns desses resultados que podem nos ajudar a pensar no fenômeno investigado neste trabalho.

\section{Fatores de influência}
A discussão sobre o escopo de quantificadores na literatura tem levantado fatores que possivelmente influenciariam na atribuição de escopo. São eles:

\begin{enumerate}
    \item Tipo de quantificador:\\
    Esse fator atribui uma \emph{hierarquia de quantificação}, inicialmente postulada por Ioup (1975), em que o quantificador mais alto na hierarquia tende a escopar sobre o quantificador mais baixo. Em geral, essa hierarquia é: \emph{each $>$ every $>$ all $>$ most $>$ many $>$ several $>$ some$_{pl}$ $>$a few}. Apesar de não obter resultados conclusivos quanto ao indefinido \emph{a}, é sugerido que ele ocuparia um lugar entre \emph{every} e \emph{all}. \citeonline{Tunstall1998} proporá uma explicação desse fator, para o caso de \emph{each} e \emph{every}, atribuindo a cada um desses quantificadores uma propriedade semântica específica.
    \item Posição gramatical:\\
    Também remetendo ao trabalho original de Ioup (1975), esse fator atribui uma \emph{hierarquia de posições gramaticais}, de modo que o quantificador que está na posição mais acima tende a escopar sobre aquele mais abaixo: \emph{tópico $>$ sujeito superficial e profundo simultaneamente $>$ sujeito superficial ou profundo $>$ objeto preposicionado $>$ objeto indireto $>$ objeto direto}.
    \item Ordem linear ou superficial: \\
    Esse fator, que, segundo Filik et al. (2004), remete aos trabalhos de Johnson-Laird (1969) e Lakoff (1971) argumenta que o quantificador mais à esquerda na ordem linear da sentença tende a escopar sobre o elemento que está mais à direita.
    \item C-comando ou posição estrutural: \\
    Esse fator diz que o elemento que c-comanda tende a tomar escopo sobre o elemento c-comandado. Refinando uma proposta de Reinhart (1976), VanLehn (1978) propõe uma hierarquia de c-comando (\emph{apud} Tunstall, 1998: 32).
\end{enumerate}

Além dessas, alguns autores (Kurtzman \& MacDonald, 1993) tratam também de uma \emph{hierarquia temática}, em que o papel temático do NP quantificado poderia ter influência na tendência a tomar escopo amplo ou não. A hierarquia seria: \emph{agente $>$ experenciador $>$ tema}. Os resultados quanto a essa influência, no entanto, parecem inconclusivos (AUTORES???). Propõem-se, também, fatores não estruturais, como papel discursivo e conhecimento de mundo, que não serão abordados neste trabalho. Cabe-nos lembrar, no entanto, já que tem correlatos estruturais na teoria linguística, a posição de Fodor (1982), que propõe que \emph{tópico} tem preferência a escopar sobre \emph{comentário}.

\section{Estruturas sintáticas}
Além do comentário sobre os fatores que supostamente influenciariam a atribuição de escopo, é preciso destacar as estruturas sintáticas utilizadas pelos diversos trabalhos revisados. São elas:

\begin{enumerate}
    \item Sentenças ativas do tipo NP V NP:\\
    Sentenças como \emph{Every kid climbed a tree} ou \emph{A kid climbed every tree} são amplamente usadas nos estudos (Kurtzman \& MacDonald, 1993; Anderson, 2004; Dwived et al., 2010).
    \item Sentenças ativas do tipo NP V NP (Pred. Sec.):\\
    Sentenças como \emph{A boy sliced every carrot thin}, usadas por Tunstall (1998)~\cite{Tunstall1998} com propósito de diferenciar as propriedades lexicais de \emph{each} e \emph{every}.
    \item Sentenças com NPs complexos do tipo SUJ. V [NP$_1$ [NP$_2$]]:\\
    Esse tipo de sentença é usada por Kurtzman \& MacDonald (1993)~\cite{KMac1993}: \emph{George has [every photo [of an admiral]]}. Nelas, o segundo NP é complemento do primeiro. Na literatura linguística, são chamadas de estruturas de \emph{inverse link} (May \& Bale, 2007; Tunstall, 1998: 86).
    \item Sentenças passivas do tipo NP V$_{Aux}$ V$_{Princ.}$ by NP:\\
    Sentenças como \emph{Every tree was climbed by a kid}. Também usadas por Kurtzman \& MacDonald (1993)~\cite{KMac1993} sem resultados conclusivos.
    \item Sentenças dativas do tipo SUJ. V OD prep. OI: \\
    Sentenças como \emph{Kelly showed a photo to every critic} são usadas por Tunstall (1998)~\cite{Tunstall1998}, Filik et al. (2004) e Paterson et al. (2006). A escolha desse tipo de sentença, e das de duplo objeto, remonta a Micham et al. (1980) e Gillen (1991).
    \item Sentenças de duplo objeto do tipo SUJ. V OI OD: \\
    Com a inversão da posição linear do objeto indireto e a queda da preposição, tem-se sentenças como \emph{Kelly showed every critic a photo}. Esse uso é feito pelos mesmos autores citados no tópico anterior.
\end{enumerate}

Alguns desses tipos podem ser logo excluídos. As sentenças com NPs complexos têm uma forte tendência a realizar escopo inverso devido a motivos estruturais. As sentenças ativas com predicados secundários são usados para objetivos específicos por Tunstall (1998)~\cite{Tunstall1998}. As sentenças passivas, no entanto, parecem demonstram com menor intensidade uma preferência de escopo (Kurtzman \& MacDonald, 1993) não obtêm preferência de leitura com elas; Catlin \& Micham (1975) e Gillen (1991) (\emph{apud} Tunstall, 1998: 81) também não encontram preferência. Sobram, então, as sentenças ativas simples, as dativas e as de duplo objeto.

Argumentaremos, com Micham et al. (1980), Gillen (1991) e Tunstall (1998)~\cite{Tunstall1998} que as ativas simples não são boas para investigação de escopo de quantificadores, visto que muitos dos fatores que possivelmente influenciariam escopo dão preferência de escopo ao NP na posição de sujeito sobre o NP na posição de objeto (é \emph{sujeito superficial} e o \emph{argumento externo}, está na primeira posição na \emph{ordem linear}, em uma posição de \emph{c-comando} e, possivelmente, está em posição de \emph{tópico discursivo} enquanto objeto está na posição de \emph{comentário}).

As sentenças dativas e de duplo objeto, no entanto, reduzem essa forte tendência. Por exemplo, nas sentenças dativas, a posição gramatical favorece escopo do objeto indireto sobre o direto (OI $>$ OD), segundo a hierarquia de Ioup (1975). A ordem linear, no entanto, ``compensa'' esse direcionamento, visto que favorece escopo do objeto direto sobre o indireto (OD $>$ OI). Esse ponto é claramente apresentado por Paterson et al. (2006)~\cite{paterson2007}.

\section{Tipos de tarefas realizadas}
Outro aspecto importante a se destacar quanto aos trabalhos estudados diz respeito ao tipo de tarefa experimental utilizada para obter os dados e justificar as análises. Basicamente, há dois grandes grupos: as tarefas \emph{off-line} e as tarefas \emph{on-line}. No primeiro grupo estão os questionários impressos realizados por Tunstall (1998)~\cite{Tunstall1998} e Anderson (2004)~\cite{Anderson2004}, mas também as perguntas de compreensão acopladas a tarefas \emph{on-line}, como as de Kurtzman \& MacDonald (1993)~\cite{KMac1993} e de Anderson (2004)~\cite{Anderson2004}. Esses resultados não refletem o processamento imediato, mas uma análise posterior, possivelmente influenciada por fatores outros, como conhecimento de mundo. Assim, cabe destacar que, apesar de realizarem tarefas \emph{on-line}, Kurtzman \& MacDonald (1993)~\cite{KMac1993} e Anderson (2004)~\cite{Anderson2004}, via de regra, não obtiveram resultados robustos advindos dessas tarefas, mas apenas dos julgamentos \emph{off-line}.

No segundo grupo, por sua vez, estão as tarefas que mensuram com mais ou menos acurácia o processamento em tempo real, independente da reflexão consciente a respeito das sentenças lidas. Os resultados de leitura automonitorada (\emph{self-paced reading}) de Tunstall (1998)~\cite{Tunstall1998} e Anderson (2004)~\cite{Anderson2004} estão nesse grupo. Mas há também métodos mais refinados, como o uso de rastreamento ocular (\emph{eye-tracking}) (Filik et al., 2004; Paterson et al., 2006) e os potenciais evocados por eventos (\emph{event-related brain potentials - ERPs}) utilizados por Dwivedi et al. (2010). Com o uso dessas técnicas tem-se uma leitura mais precisa quanto ao funcionamento do processador humano.

Quanto às tarefas \emph{off-line}, cabe destacar os diversos tipos de perguntas utilizadas. Kurtzman \& MacDonald (1993)~\cite{KMac1993} se valem de uma pergunta relacionada à sentença continuativa que questionava se ela ``fazia sentido'' (\emph{made sense}) e era uma ``continuação natural'' (\emph{natural continuation}) da sentença quantificada. Tunstall (1998)~\cite{Tunstall1998} valia-se de uma tarefa de ``parar de fazer sentido'' (\emph{stop-making sense}), na qual os participantes interrompiam a leitura automonitorada assim que achavam que a frase parava de fazer sentido. Anderson (2004)~\cite{Anderson2004}, apesar de usar uma escala linear de 1 a 5 em um dos experimentos, em que 1 mostrava uma continuação singular e 5 uma continuação plural e o participante deveria marcar qual a mais plausível, em geral valeu-se de uma pergunta do tipo ``quantos'' (\emph{how many}), a que os participantes poderiam responder com ``um'' (\emph{one}) ou ``vários'' (\emph{several}). Filik et al. (2004)~\cite{filik2004} e Paterson et al. (2006)~\cite{paterson2007} investigam suas sentenças quantificadas previamente aos experimentos em si, apresentando-as a participantes que as julgavam em uma escala de 1 a 5 como se referindo ``definitivamente a uma entidade'' (\emph{definitely one}) ou ``definitivamente a mais de uma entendidade'' (\emph{definitely more then one}). No caso do estudo com \emph{each}, os autores encontram uma diferença entre as ordem \emph{a-each} e \emph{each-a} nas sentenças de duplo objeto, mas não encontram com as dativas, argumentando, portanto, que não há viés para uma ou outra leitura com essas sentenças (Paterson et al., 2006: 6).

A tabela~\ref{tab:t1} apresenta um resumo dos resultados obtidos quanto à preferência de interpretação dado o tipo de tarefa experimental realizado (uma revisão completa dos experimentos será feita no trabalho final, estando além do escopo deste projeto).

\begin{table}[h]
    \centering
    \begin{tabular}{lll}
    \toprule
         & a-every & every-a  \\ [.3em]
         \midrule
         off-line & 7(8) & 7(7) \\
         on-line & 4(6) & 2(7) \\
    \bottomrule
    \end{tabular}
    \caption{Preferência por escopo linear nos experimentos revisados}
    \label{tab:t1}
\end{table}

Conforme a tabela~\ref{tab:t1}, os estudos parecem indicar que:

\begin{enumerate}[label=(\roman*)]
    \item a preferência por escopo linear ocorre tanto para ordem \emph{a-every} quando para a ordem \emph{every-a} quando a tarefa é \emph{off-line}, quando o que se está mensurando demanda uma reflexão consciente, possivelmente metalinguística, sobre a sentença lida;
    \item a preferência por escopo linear, todavia, não ocorre ou ocorre em menor escala quando a tarefa é \emph{on-line}, ou seja, quando o que se está mensurando não demanda reflexão consciente, dependendo exclusivamente do processamento em tempo real.
\end{enumerate}

Esses dados parecem estar de acordo com os apontamentos de Rodrigues \& Marcilese (2014) sobre o português listados ao final do capítulo anterior.

\section{Competição ou subespecificação}
O terceiro ponto a se destacar nos estudos é o embate entre a ideia de \emph{competição de representações} e \emph{subespecificação}, que discutiremos aqui estritamente vinculado aos resultados obtidos com as diferentes ordens lineares dos quantificadores (\emph{a-every} e \emph{every-a}).

Em geral, os resultados mostram uma preferência por escopo linear (\emph{continuação singular}) quando a ordem dos quantificadores é \emph{a-every} (Kurtzman \& MacDonald, 1993; Tunstall, 1998; Anderson, 2004, Filik et al., 2004~\cite{filik2004}). Quando a ordem é \emph{every-a}, no entanto, os resultados divergem. Kurtzman \& MacDonald (1993)~\cite{KMac1993} encontram uma tendência para escopo linear (\emph{continuação plural}), mas dizem que é uma tendência menos robusta do que para \emph{a-every}. Tunstall (1998)~\cite{Tunstall1998} e Dwivedi et al. (2010)~\cite{Dwivedi2010} não encontram essa preferência. Apenas Anderson (2004)~\cite{Anderson2004} tem resultados que parecem indicar uma preferência pelo escopo linear, seja com contexto prévio (experimento 3) seja em frases isoladas (experimento 4). Ver também os resultados do experimento 5.

A fim de explicar os resultados mais robustos com a ordem \emph{a-every}, Kurtzman \& MacDonald (1993: 257)~\cite{KMac1993}, seguindo proposta de Fodor (1982), propõem o princípio de referência unitária (\emph{single reference principle}), segundo o qual o processador, ao encontrar o artigo indefinido \emph{a} na posição de sujeito, imediatamente atribui a ele referência a uma única entidade. Assim, apenas quando encontra o quantificador universal \emph{every} na posição de objeto, é que surge a possibilidade de mudar de referência. Esse processo de mudança, no entanto, é custoso, de modo que o processador evitará fazê-lo, explicando a maior tendência de escopo linear (\emph{a $>$ every}). Filik et al. (2004)~\cite{filik2004} testam essa proposição com uso de técnica de rastreamento ocular. Obtendo maiores tempos para a ordem linear \emph{a-every} do que para \emph{every-a}, argumentam que têm indício de que de fato o que ocorre é um processo de revisão da representação até então construída.

Tal explicação, contudo, não é unânime. Tunstall (1988: 70)~\cite{Tunstall1998}, ao não obter preferência de escopo com a ordem linear \emph{every-a}, propõe o princípio da vagueza (\emph{vagueness principle}), segundo o qual o processador, ao se deparar com \emph{every} na primeira posição, mantém a sua referência vaga (ou subespecificada) quanto ao número de elementos envolvidos, que deverá ser definida de acordo com o contexto subsequente. Já que a representação é deixada em aberto, tanto a continuação plural (escopo linear) quanto a continuação singular (escopo inverso) são possíveis, de modo que não há preferência por nenhuma delas. Dwivedi et al. (2010)~\cite{Dwivedi2010} também defendem esse ponto, visto que seus resultados de potenciais evocados (ERPs) não apresentam N400 ou P600, que indicariam revisão das representações previamente estabelecidas. Em vez de uma competição entre representações, propõe-se uma subespecificação de representações.
