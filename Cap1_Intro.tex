\chapter{Introdução}
Este projeto está vinculado à \emph{Linha 2 -- Língua e Cognição: representação, processamento e aquisição da linguagem} do Programa de Pós--Graduação em Estudos da Linguagem, inserindo-se nos trabalhos desenvolvidos no âmbito do Laboratório de Psicolinguística e Aquisição da Linguagem da Pontifícia Universidade Católica do Rio de Janeiro (LAPAL/PUC-Rio).

O trabalho busca investigar a atuação do processador humano durante a compreensão de sentenças ambíguas que contenham dois quantificadores, a saber: o quantificador universal \emph{todo} e o quantificador indefinido \emph{um}, seja quando o universal aparece primeiro na ordem linear da frase seja quando essa ordem se inverte.

\begin{enumerate}
  \item O vendedor mostrou \textbf{toda} pulseira de ouro para \textbf{um} comprador.
  \item O vendedor mostrou \textbf{uma} pulseira para \textbf{todo} comprador de joias.
\end{enumerate}

Na teoria linguística, o estudo sobre a relação entre quantificadores tem demonstrado (REFERÊNCIAS) que a interpretação do substantivo antecedido pelo indefinido (nos casos acima, os nomes \emph{comprador} em (1) e \emph{pulseira} em (2)) pode receber pelo menos duas interpretações: a de que haveria apenas um único comprador e uma única pulseira; e a de que haveria mais de um comprador e mais de uma pulseira, ou seja, a sentença (1) poderia ser parafraseada por \emph{para cada comprador em questão, o vendedor mostrou todas as pulseiras}.
Essa segunda leitura é denominada \emph{leitura distributiva} enquanto a primeira é denominada \emph{leitura coletiva}. Tais leituras seriam possíveis, segundo a teoria linguística, devido à relação estabelecida, em cada caso, entre os quantificadores. Quando o primeiro quantificador na ordem linear determina o escopo, diz-se que que se tem \emph{escopo linear}; quando o segundo quantificador determina o escopo, então diz-se que se tem \emph{escopo inverso} ou \emph{escopo invertido}. Um outro modo é dizer que o quantificador que determina o escopo recebeu, na sentença em questão, \emph{escopo amplo} enquanto o que não determina recebeu \emph{escopo restrito}.

Esse tipo de fenômeno, amplamente documentado e estudado na teoria linguística desde os anos 1970, pelo menos (REFERÊNCIAS), tanto no português quanto em outras línguas, se mostra muito importante para a teoria Psicolinguística, visto que permite investigar o modo como o processador linguístico humano ou \emph{parser} lida com sentenças que são ambíguas. Uma vez que a compreensão humana é robusta (REFERÊNCIAS) e potencialmente guiada por fatores estruturais do \emph{input} linguístico disponível aos ouvintes (REFERÊNCIAS), é preciso saber como ele resolve sentenças ambíguas e chega a uma única interpretação para elas. Fenômenos como o escopo de quanticadores, portanto, permitem-nos investigar os fatores (estruturais, semânticos, pragmáticos, contextuais) que levam o processador a tomar determinadas decisões interpretativas e, além disso, permitem a formulação de teorias sobre a própria atuação do \emph{parser} humano no tratamento de sentenças, informando, por exemplo, se há uma leitura padrão ou \emph{default} ou se as leituras permanecem subespecificadas ou vagas até que informação de natureza linguística e/ou contextual permita decidir sobre uma delas. Esse trabalho poderia, assim, contribuir para o refinamento de tais posições teóricas.

Além disso, como bem demonstrado por Barcellos (2017)~\cite{Barcellos2017}, a possibilidade de sentenças apresentarem tanto leituras coletivas quanto distributivas pode ter impactos no ensino. Alguns enunciados de questões matemáticas nos anos iniciais, segundo aquela autora, tornam-se ambíguos sobretudo devido à estrutura linguística possibilitar ambas as interpretações. Um melhor entendimento sobre o processamento de sentenças que permitem tais interpretações, portanto, pode auxiliar a evitar essas construções em contextos pedagógicos em que gerariam confusões.

Por fim, esse tipo de pesquisa pode ajudar a lançar alguma luz sobre o funcionamento de um aspecto da cognição humana que ainda precisa de muito esclarecimento, a saber: o processamento, em tempo real, de expressões linguísticas ambíguas.
