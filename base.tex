\documentclass[12pt,a4paper,oneside,english,brazil]{abntex2}

\usepackage[utf8]{inputenc}
\usepackage{booktabs}
\usepackage{microtype}
\usepackage{epigraph}
\usepackage{csvsimple}
\usepackage{amsmath}
\usepackage{index}
\usepackage{graphicx}
\usepackage{ragged2e}
\usepackage{xcolor}
\usepackage{enumitem}
\usepackage{moreenum}
\usepackage[alf]{abntex2cite} % será aposentado num futuro próximo;
                              % precisamos conversar sobre o biblatex, mas
                              % por ora usa esse mesmo

\newcommand{\eng}[1]{\foreignlanguage{english}{\emph{#1}}}

\numberwithin{figure}{chapter}
\numberwithin{table}{chapter}

\titulo{Estruturas distributivas em Português Brasileiro: aspectos linguísticos e processuais}
\autor{Igor Costa}
\data{\today}

\begin{document}
\frenchspacing
\pretextual
\newpage
\imprimircapa

\newpage
\tableofcontents

\newpage
\textual
\chapter{Introdução}
Este projeto está vinculado à \emph{Linha 2 -- Língua e Cognição: representação, processamento e aquisição da linguagem} do Programa de Pós--Graduação em Estudos da Linguagem, inserindo-se nos trabalhos desenvolvidos no âmbito do Laboratório de Psicolinguística e Aquisição da Linguagem da Pontifícia Universidade Católica do Rio de Janeiro (LAPAL/PUC-Rio).

O trabalho busca investigar a atuação do processador humano durante a compreensão de sentenças ambíguas que contenham dois quantificadores, a saber: o quantificador universal \emph{todo} e o quantificador indefinido \emph{um}, seja quando o universal aparece primeiro na ordem linear da frase seja quando essa ordem se inverte.

\begin{enumerate}
  \item O vendedor mostrou \textbf{toda} pulseira de ouro para \textbf{um} comprador.
  \item O vendedor mostrou \textbf{uma} pulseira para \textbf{todo} comprador de joias.
\end{enumerate}

Na teoria linguística, o estudo sobre a relação entre quantificadores tem demonstrado (REFERÊNCIAS) que a interpretação do substantivo antecedido pelo indefinido (nos casos acima, os nomes \emph{comprador} em (1) e \emph{pulseira} em (2)) pode receber pelo menos duas interpretações: a de que haveria apenas um único comprador e uma única pulseira; e a de que haveria mais de um comprador e mais de uma pulseira, ou seja, a sentença (1) poderia ser parafraseada por \emph{para cada comprador em questão, o vendedor mostrou todas as pulseiras}.
Essa segunda leitura é denominada \emph{leitura distributiva} enquanto a primeira é denominada \emph{leitura coletiva}. Tais leituras seriam possíveis, segundo a teoria linguística, devido à relação estabelecida, em cada caso, entre os quantificadores. Quando o primeiro quantificador na ordem linear determina o escopo, diz-se que que se tem \emph{escopo linear}; quando o segundo quantificador determina o escopo, então diz-se que se tem \emph{escopo inverso} ou \emph{escopo invertido}. Um outro modo é dizer que o quantificador que determina o escopo recebeu, na sentença em questão, \emph{escopo amplo} enquanto o que não determina recebeu \emph{escopo restrito}.

Esse tipo de fenômeno, amplamente documentado e estudado na teoria linguística desde os anos 1970, pelo menos (REFERÊNCIAS), tanto no português quanto em outras línguas, se mostra muito importante para a teoria Psicolinguística, visto que permite investigar o modo como o processador linguístico humano ou \emph{parser} lida com sentenças que são ambíguas. Uma vez que a compreensão humana é robusta (REFERÊNCIAS) e potencialmente guiada por fatores estruturais do \emph{input} linguístico disponível aos ouvintes (REFERÊNCIAS), é preciso saber como ele resolve sentenças ambíguas e chega a uma única interpretação para elas. Fenômenos como o escopo de quanticadores, portanto, permitem-nos investigar os fatores (estruturais, semânticos, pragmáticos, contextuais) que levam o processador a tomar determinadas decisões interpretativas e, além disso, permitem a formulação de teorias sobre a própria atuação do \emph{parser} humano no tratamento de sentenças, informando, por exemplo, se há uma leitura padrão ou \emph{default} ou se as leituras permanecem subespecificadas ou vagas até que informação de natureza linguística e/ou contextual permita decidir sobre uma delas. Esse trabalho poderia, assim, contribuir para o refinamento de tais posições teóricas.

Além disso, como bem demonstrado por Barcellos (2017)~\cite{Barcellos2017}, a possibilidade de sentenças apresentarem tanto leituras coletivas quanto distributivas pode ter impactos no ensino. Alguns enunciados de questões matemáticas nos anos iniciais, segundo aquela autora, tornam-se ambíguos sobretudo devido à estrutura linguística possibilitar ambas as interpretações. Um melhor entendimento sobre o processamento de sentenças que permitem tais interpretações, portanto, pode auxiliar a evitar essas construções em contextos pedagógicos em que gerariam confusões.

Por fim, esse tipo de pesquisa pode ajudar a lançar alguma luz sobre o funcionamento de um aspecto da cognição humana que ainda precisa de muito esclarecimento, a saber: o processamento, em tempo real, de expressões linguísticas ambíguas.

\chapter{Sobre o \emph{todo} no português}
O termo \emph{distributividade} pode ser definido como ``a atribuição de um predicado \textbf{a membros} de um NP [sintagma nominal, do inglês \emph{Nominal Phrase}] complexo e não ao NP complexo como um todo'' (Clifton \& Frazier, 2012)~\cite{ClifFrazier2012}. Em português, a estrutura linguística mais destacada para expressar relações de distributividade são os chamados \emph{operadores distributivos universais}, a saber: \emph{todo} e \emph{cada} (Negrão, 2002: 186-187)~\cite{neg2002}. Embora o uso de \emph{cada} seja reconhecidamente um uso distributivo, o operador \emph{todo} apresenta algumas nuances, sobretudo porque pode aparecer em pelo menos três formas distintas: (i) todo + DP plural; (ii) todo + DP singular; e (iii) todo + NP singular, conforme abaixo:

\begin{enumerate}
  \item \textbf{Todas as crianças} jogam bola.
  \item \textbf{Toda a criançada} joga bola.
  \item \textbf{Toda criança} joga bola.
\end{enumerate}

Como veremos em seguida, há indícios na literatura (Negrão, 1999 \& 2002~\cite{neg2002}; Silva, 2012~\cite{silva2012}; Rodrigues \& Marcilese, 2014~\cite{RodMarc2014}; Pires de Oliveira, 2003~\cite{PiresOliv2003}) de que esse elemento parece apresentar um comportamento errático, ora permitindo leituras coletivas ora distributivas ou mesmo genéricas.

\section{\emph{Todo} como um indefinido}
O trabalho de Negrão (2002)~\cite{neg2002} contrasta o comportamento de \emph{todo} com \emph{cada}, mostrando que, embora aparentemente apresentem comportamento semelhante, permitindo leituras distributivas, essa semelhança é apenas superficial. Isso ocorre, dentre outros motivos,  porque \emph{todo} apresenta uma leitura genérica em muitos contextos, como mostram as sentenças a seguir.

\begin{enumerate}[resume]
    \item *Cada homem é inteligente.
    \item Todo homem é inteligente.
\end{enumerate}

A última sentença não afirma que há uma série de subeventos de \emph{ser inteligente} ou que a propriedade ``inteligência'' é distribuída numa relação uma a uma com os homens. Ela diz que, em geral, genericamente, os homens são inteligentes. Essa leitura genérica também é vista quando \emph{todo+NP} não ocupa a posição de sujeito:

\begin{enumerate}[resume]
    \item *Aquele médico examinou todo paciente.
    \item *Aquele médico examinou todo paciente num horário.
 \end{enumerate}

No caso dessas sentenças, \emph{todo} não poderia subir em Forma Lógica (LF, do inglês \emph{Logical Form}) acima do VP (sintagma verbal, do inglês \emph{Verbal Phrase}) para escopar sobre o evento de \emph{examinar}. Mesmo com o acréscimo de um adjunto locativo-temporal (\emph{num horário}), que salvaria sentenças com \emph{cada} nesse contexto (vide a gramaticalidade de \emph{Aquele médico examinou cada paciente num horário}), a leitura distributiva ainda não se realiza. O único sentido possível para as sentenças acima seria aquele parafraseado por \emph{todos os pacientes} ou \emph{os pacientes como um grupo, um conjunto}.

Esse argumento se torna ainda mais forte devido ao fato de que \emph{todo+NP} não toma escopo sobre um agente da passiva, mesmo quando este vem acompanhado de um termo como \emph{diferente}, que garantiria a leitura distributiva.

 \begin{enumerate}[resume]
     \item ?Todo texto foi lido por um aluno diferente.
 \end{enumerate}

Nesse caso, segundo a autora, a sentença não recebe uma leitura distributiva, mas uma em que \emph{diferente} é lido como \emph{fora do comum}, \emph{excepcional}, ou seja, uma leitura adjetival de \emph{todo}.

Por fim, ainda há o argumento de que \emph{todo+NP} não sustenta leitura de escopo invertido quando fora do sujeito em sentenças como aquela abaixo, em que a leitura possível pode ser parafraseada por \emph{Maria tem um só vestido, o mesmo vestido, e o usa em todas as ocasiões} e não que ela teria \emph{vários vestidos, um para cada ocasião}, ou seja, uma leitura distributiva. Esse tipo de fenômeno ocorreria porque \emph{todo} não consegue subir em LF para uma posição acima de \emph{um vestido}, realizando escopo invertido.

\begin{enumerate}[resume]
    \item Maria tem um vestido para toda ocasião.
\end{enumerate}

Com isso, a autora afirma que \emph{todo+NP} só tem leitura distributiva em posição de sujeito de sentenças ativas, e que a leitura coletiva é proibida em outros contextos, já que ele não conseguiria realizar escopo invertido. Quando na posição de objeto, a leitura de \emph{todo+NP} tem de ser genérica. Ela argumenta, ainda, que essa expressão não seria, em PB, um verdadeiro quantificador, mas um indefinido.

Em trabalho posterior (Müller, Negrão \& Quadros Gomes, 2007~\cite{MullerETal2007}), a autora parece reformular a sua proposta original, argumentando que \emph{todo} é de fato um quantificador universal distributivo em todas as circunstâncias em que ocorre, seja acompanhado de NP singular (\emph{todo+NP}) seja acompanhado de DP singular ou plural (\emph{todo+DP}), não sendo ambíguo entre uma interpretação coletiva e distributiva. Na verdade, segundo elas, a leitura coletiva com \emph{todo + DP} é apenas uma ilusão. Essa ilusão adviria do fato de \emph{todo} poder quantificar sobre um nome nu (\emph{todo+NP}) ou uma descrição definida (\emph{todo+DP}). As diferentes interpretações adviriam, nessa visão, do tipo de elemento que acompanha o quantificador e não de uma propriedade inerente a ele ou ao fato de ele não ser um quantificador verdadeiro.

\section{\emph{Todo} como um quantificador modal}
Em contraste com a visão de que \emph{todo}, no português brasileiro, seria um indefinido ou que as suas diferentes leituras adviriam do fato de esse elemento poder ser acompanhado de NP ou de DP, Pires de oliveira (2003)~\cite{PiresOliv2003} defende que \emph{todo+NP} é, na verdade, um \emph{quantificador modal}, ou seja, que ele requer em seu escopo a presença de um elemento de natureza modal, comportando-se como o \emph{any} do inglês. A fim de defender esse ponto, a autora elenca algumas propriedades de \emph{todo+NP}. Abaixo revisamos a maioria dessas propriedades, visto que elas são de sobremaneira relavantes para a construção dos estímulos experimentais que utilizaremos neste estudo. Os exemplos dados, a menos que especicificado em contrário, são originais da autora.

\subsection{Flutuação}
A primeira das propriedades de \emph{todo+NP} que abordaremos é aquela chamada por Pires de Oliveira (2003)~\cite{PiresOliv2003} de \emph{flutuação}. Enquanto o quantificador universal \emph{todo+DP} pode ser trocado de posição em sentenças como as de (1), o mesmo não ocorre com \emph{todo+NP} em (2):

\begin{enumerate}
    \item a. Todas as crianças choram. \\
    b. As crianças todas choram.
    \item a. Toda criança chora. \\
    b. *Criança toda chora.
\end{enumerate}

\subsection{Retomada anafórica}
A segunda das propriedades de \emph{todo+NP} é aquela que contrasta a retomada do elemento nominal por um pronome anafórico, como nos exemplos abaixo:

\begin{enumerate}[resume]
    \item As crianças, todas elas choram.
    \item *A criança, toda ela chora.
\end{enumerate}

Isso ocorreria, segundo Pires de Oliveira (2003)~\cite{PiresOliv2003} porque \emph{todo+NP} se combinaria com um predicado, enquanto \emph{todo+DP} se combinaria com um argumento e, sendo argumento, pode se sustentar sozinho, sem a necessidade de ser ligado (\emph{bounded}) por um operador. Como a retomada anafórica só retoma indivíduos (ou seja, argumentos e não predicados), então ficaria explicado por que ela não pode ocorrer com \emph{todo+NP}. Esse mesmo motivo explicaria a diferença no caso de \emph{flotation} dado acima.

\subsection{Contextos episódicos}
Outra importante propriedade de \emph{todo+NP} está relacionada à leitura episódica ou não episódica. Observe os exemplos abaixo, dados em Pires de Oliveira (2003:367)~\cite{PiresOliv2003}:

\begin{enumerate}[resume]
    \item Toda a criançada chora. \\
    Trata de um grupo específico de crianças; \\
    Faz leitura episódica de \emph{chorar}.
    \item Toda criança chora. \\
    Trata de uma leitura genérica de \emph{chorar}; descreve uma ``lei'' sobre comportamento de crianças; \\
    Faz leitura não episódica de \emph{chorar}.
\end{enumerate}

Esse tipo de fenômeno ocorreria porque \emph{todo+DP}, por ser uma expressão definida, trata de um indivíduo particular. Esse indivíduo, no entanto, pode ser tanto \emph{atômico} quanto \emph{grupal}. Sendo \emph{atômico}, a sentença tem apenas uma leitura; sendo \emph{grupal}, a sentença é ambígua, podendo ter leitura \emph{contínua} (conta indivíduos de um conjunto) ou \emph{descontínua} (toma o indivíduo em sua totalidade), no sentido de Peres (1992). Veja a distinção nas sentenças abaixo, retiradas de Pires de Oliveira (2003: 367)~\cite{PiresOliv2003}:

\begin{enumerate}[resume]
    \item O menino todo se machucou. \\
    O mesmo que \emph{o menino inteiro se machucou} \\
    Única leitura possível: contínua.
    \item Toda a criançada chora. \\
    Pode tanto ter leitura contínua quanto a descontínua (p. 367), significando:
    \emph{O conjunto inteiro de crianças chora} ou \emph{cada criança do conjunto chora}.
\end{enumerate}

\emph{Todo+NP}, no entanto, não tem essa propriedade, nunca podendo aparecer em contexto episódico:

\begin{enumerate}[resume]
    \item *Toda a criança se machucou. \\
    \textcolor{red}{Erica, acho que a Roberta se equivocou aqui (exemplo 18, página 368... Deveria ser \emph{*Toda criança}, não?! A ideia aqui acho é a frase só poder ser lida em sentido genérico (é uma ``lei'' que \emph{toda criança se machuca}, mas não episódica, leitura descontínua, \emph{a criança se machucou}... a flexão do verbo impede a leitura genérica e a frase é ruim por isso).}
\end{enumerate}

A impossibilidade de aspecto progressivo é outra propriedade que demonstra a não ocorrência em contexto expisódico, visto que aspecto progressivo só pode ocorrer com definidos, ou seja, com \emph{todo+DP}, mas não com \emph{todo+NP}.

\begin{enumerate}[resume]
    \item Toda a criançada está brincando.
    \item *Toda criança está brincando. \\
    Essa sentença não pode ser interpretada no sentido de \emph{cada uma das crianças está brincando}, ou seja, com leitura descontínua.
\end{enumerate}

O mesmo pode ser dito sobre a impossibilidade de combinação de \emph{todo+NP} com partitivos, enquanto os sintagmas definidos podem fazê-lo explicitamente:

\begin{enumerate}[resume]
    \item Toda a criançada da festa se machucou.
    \item ?Toda criança da festa se machucou.
\end{enumerate}

\subsection{\emph{Subtrigging}}
Um outra propriedade de \emph{todo+NP} é o fato de eles poderem ocorrer na posição de objeto (ao contrário do que argumenta Negrão, 2002) desde que o nome nu esteja sendo modificado (implícita ou explicitamente), visto que \emph{todo+NP} não é sobre indivíduos particulares, mas ``sobre indívudos possíveis que podem ser caracterizados pela propriedade expressa pelo substantivo comum.'' (Pires de Oliveira, 2003: 370)~\cite{PiresOliv2003}. Por exemplo:

\begin{enumerate}[resume]
    \item Ele canta todas as canções. \\
    Trata de canções particulares: canção A, B, C, etc. Logo, é gramatical.
    \item *Ele canta toda canção. \\
    Trata da totalidade das canções possíveis. Logo, não pode ser gramatical.
    \item Ele canta toda canção que escuta no Spotify. (Exemplo meu.) \\
    Trata não do conjunto de canções possíveis, mas do conjunto de canções possíveis que foram individualmente ouvidas.
\end{enumerate}

Esse argumento é usado pela autora para descaracterizar a posição de Negrão (2002)~\cite{neg2002}, de que \emph{todo+NP} não poderia tomar escopo invertido em posição de objeto. Na verdade ele pode, desde que o nome que o acompanha esteja modificado. De acordo com essa autora, a última sentença acima não poderia ser licenciada porque haveria um conflito entre o traço [+universal] do quantificador \emph{todo} e o traço [+existencial] do predicado verbal.

Essa ideia está vinculada à proposta da autora de que tal elemento é um indefinido e não um quantificador. A possibilidade de \emph{subtrigging}, entretanto, desafia essa visão, visto que, \emph{todo+NP} não se comporta como um, como mostra o exemplo:

\begin{enumerate}[resume]
    \item Ele canta uma canção que escuta no Spotify. \\
    Essa frase só tem leitura existencial, não genérica, como é o caso da mesma sentença com \emph{toda} no lugar de \emph{uma}.
\end{enumerate}

\subsection{Licenciamento como sujeito}
Outro contraste interessante de \emph{todo+NP}, esse também apontado por Negrão (2002)~\cite{neg2002}, é o fato de esse elemento poder ser licenciado como sujeito quando o objeto é um sintagma quantificador de grupo:

\begin{enumerate}[resume]
    \item *Cada homem ama aquela mulher. \\
    Por ser integralmente distributivo \emph{cada} torna a sentença agramatical.
    \item Todo homem ama aquela mulher. \\
    Essa sentença mostra que \emph{todo+NP} não pode ser total ou somente distributivo.
\end{enumerate}

O mesmo vale para o fato, também apontado por Negrão (2002)~\cite{neg2002}, de que \emph{todo+NP} pode ser sujeito de predicados \emph{individual level}, com leitura genérica, na forma de ``lei'', enquanto o distributivo ``puro'' não pode:

\begin{enumerate}[resume]
    \item *Cada homem é inteligente.
    \item Todo homem é inteligente.
\end{enumerate}

\subsection{Modificadores}
Assim como \emph{any}, que aceita \emph{almost} e \emph{absolutely} como modificadores, \emph{todo+NP} aceita \emph{quase} e \emph{certamente} como modificadores, uma propriedade dos universais, mas não dos indefinidos. Além disso, também como \emph{any}, aceitam \emph{exceptional phares}, outra propriedade dos universais, mas não dos indefinidos.

\begin{enumerate}[resume]
    \item Quase todo menino brinca.
    \item Certamente todo menino brinca.
    \item Todo menino exceto o João brinca.
\end{enumerate}

No caso das sentenças acima, todas recebem leitura genérica, na forma de ``lei''.

\subsection{Estrutura tripartida}
Outra propriedade de \emph{todo+NP} que parece colocá-lo no grupo dos quantificadores é o fato de engendrarem estrutura tripartida.

\begin{enumerate}[resume]
    \item *João conversou com todo aluno \textbf{bravo}.
    \item João conversou com todo aluno \textbf{que estava bravo}.
\end{enumerate}

Na primeira sentença, a única leitura possível seria aquele em que \emph{todo aluno bravo possível} teve uma conversa com João. No entanto, essa leitura é impossível, visto que é uma sentença episódica (\emph{conversou}) e, como tal, exigiria uma situação particular. No caso da segunda sentença isso pode ocorrer, visto que a oração relativa restringe a uma situação particular.

\subsection{A natureza modal de \emph{todo+NP}}
Em resumo, visto que \emph{todo+NP} se comporta como um quantificador e não como um indefinido, não apresentando as propriedades deste, a autora defende que ele de fato é um quantificador, mas um quantificador excepcional, visto que requer um elemento de natureza modal em seu escopo. Um primeiro argumento nessa linha são as implicações possíveis com \emph{todo+NP} que se perdem quando ele é substituído por um indefinido:

\begin{enumerate}[resume]
    \item Ela pode cantar toda canção desse álbum. \\
    \emph{Aquarela do Brasil} é uma canção desse álbum. \\
    Logo: Ela pode cantar \emph{Aquarela do Brasil}.
    \item Ela pode cantar uma canção desse álbum. \\
    Nesse caso, a leitura genérica se perde e a implicação acima se torna falsa. \\
    Isso provaria que a qualidade quantificacional é uma propriedade do \emph{todo} e não do contexto morfológico.
\end{enumerate}

Um segundo argumento se vincula à possibilidade de \emph{todo+NP} poder tomar escopo inverso na posição de objeto, já discutido acima. Ao contrário de Negrão (2002)~\cite{neg2002}, que afirma ter a sentença abaixo apenas uma leitura, Pires de Oliveira (2003)~\cite{PiresOliv2003} argumenta que ela tem duas leituras possíveis:

\begin{enumerate}[resume]
    \item Maria tem um vestido para toda ocasião. \\
    Negrão (2002)~\cite{neg2002}: escopo invertido é impossível. \\
    Única leitura permitida é aquela em que Maria tem um único vestido e o usa em diferentes ocasiões.\\
    \\
    Pires de Oliveira (2003)~\cite{PiresOliv2003}: é possível haver uma leitura distributiva (um vestido diferente para cada ocasião).
\end{enumerate}

Dado, então, que \emph{todo+NP} seria melhor classificado como um quantificador e não como um indefinido, a autora aponta dois possíveis modos de facilitar a leitura modal, a saber: (i) o verbo portar uma pressuposição que pode ser acomodada no restritor do operador; e (ii) o NP ser modificado a fim de prover material para atuação do operador. Essa seria uma visão diferente para \emph{subtrigging}, visto que esse modificador proveria material para permitir a partição da sentença.

\section{Resultados experimentais em português}
Rodrigues \& Marcilese (2014)~\cite{RodMarc2014} buscam fazer uma revisão do tipo de resultado experimental obtido com \emph{todo+NP} (entre outros quantificadores) em contextos experimentais. O que elas reportam é que o tipo de tarefa experimental parece ter um impacto no modo como \emph{todo} é preferencialmente interpretado.

\begin{enumerate}
    \item Tarefa de seleção de imagens \\
    Leitura preferencial: coletiva \\
    Médias de 4,05 em escala de 6 ponto
    \item Tarefa de adequação sentença-imagem \\
    Leitura preferencial: Não foi possível aferir preferência \\
    86\% de leituras coletivas X 56\% de leituras distributivas
    \item Tarefa de adequação sentença-imagem evitando pareamento 1 a 1 \\
    Leitura preferencial: coletiva \\
    75\%  de leituras coletivas X 52,2\%  de leituras distributivas
    \item Tarefa de produção de imagem com interpretação livre \\
    Leitura preferencial: ??? \\
    42\% de leituras coletivas (ou genéricas?) X 47\% de distributivas X 11\% de outras
    \item  Tarefa de leitura automonitorada \\
    Leitura preferencial: coletiva \\
    Maiores tempos de reação para leitura distributiva
    \item Tarefa de Leitura automonitorada com julgamento de gramaticalidade \\
    Leitura preferencial: coletiva \\
    47,5\% de leituras coletivas X 19\% de distributivas para julgamento \emph{SIM, é sentença aceitável}
    \item Tarefa de Julgamento de gramaticalidade com resposta escalar (1-ruim a 5-bom) \\
    Leitura preferencial: Não foi possível aferir preferência \\
    Coletiva não se acumulou no topo da escala (julgamentos 4 e 5) X distributiva não se acumulou em nenhum ponto da escala
    \item Tarefa de julgamento e Tempo de Reação diante de contexto prévio \\
    Leitura preferencial: distributiva \\
    Contexto distributivo recebeu mais julgamentos \emph{SIM} \& contexto coletivo teve maiores tempos para julgamento \emph{NÃO}
    \item Tarefa de julgamento e Tempo de Reação diante de contexto prévio com ranqueamento \\
    Leitura preferencial: Não foi possível aferir preferência \\
    \emph{Todo+NP} nunca é escolhido como primeira ou última opção
\end{enumerate}

Diante de tais resultados, as autoras excluem a possibilidade de \emph{todo+NP} não permitir leituras coletivas: mesmo que não pareça haver uma leitura preferencial com esse tipo de expressão, a leitura distributiva aparece como uma possibilidade em muitos dos casos. Ademais, dependendo da tarefa experimental, até mesmo leituras coletivas podem não ser as preferidas, como no caso de maiores tempos para rejeição de figuras com leitura coletiva em experimento das próprias autoras. Além das leituras distributivas aparecerem com \emph{todo+NP}, as autoras afirmam também que vários estudos mostram outros tipos de leituras com essa expressão, como leituras adverbiais e genéricas.

Desse modo, elas afirmam que \emph{todo+NP} deve ser avaliado como um elemento de natureza indeterminada, nos moldes do que propõe Negrão (2002)~\cite{neg2002}, sendo um item subespecificado.

\chapter{Resultados de experimentos em inglês}

\textcolor{red}{Teste de bibliografia:\\
Kurtzman \& MacDonald (1993)~\cite{KMac1993}~\cite{KMac1993}\\
Tunstall (1998)~\cite{Tunstall1998}~\cite{Tunstall1998}\\
Anderson (2004)~\cite{Anderson2004}~\cite{Anderson2004}\\
Paterson et al. (2007)~\cite{paterson2007}\\}

Quando se trata da verificação das relações de escopo, a literatura em língua inglesa apresenta uma série de trabalhos que investiga o comportamento do processador humano em sentenças duplamente quantificadas. Os resultados, no entanto, são muitas vezes conflitantes, mesmo quando se olha apenas para a relação entre o quantificador universal \emph{every} -- que, semanticamente, é o elemento que mais se aproxima do \emph{todo} em português -- e o indefinido \emph{a}. Nesta seção, faremos uma revisão de alguns desses resultados que podem nos ajudar a pensar no fenômeno investigado neste trabalho.

\section{Fatores de influência}
A discussão sobre o escopo de quantificadores na literatura tem levantado fatores que possivelmente influenciariam na atribuição de escopo. São eles:

\begin{enumerate}
    \item Tipo de quantificador:\\
    Esse fator atribui uma \emph{hierarquia de quantificação}, inicialmente postulada por Ioup (1975), em que o quantificador mais alto na hierarquia tende a escopar sobre o quantificador mais baixo. Em geral, essa hierarquia é: \emph{each $>$ every $>$ all $>$ most $>$ many $>$ several $>$ some$_{pl}$ $>$a few}. Apesar de não obter resultados conclusivos quanto ao indefinido \emph{a}, é sugerido que ele ocuparia um lugar entre \emph{every} e \emph{all}. \citeonline{Tunstall1998} proporá uma explicação desse fator, para o caso de \emph{each} e \emph{every}, atribuindo a cada um desses quantificadores uma propriedade semântica específica.
    \item Posição gramatical:\\
    Também remetendo ao trabalho original de Ioup (1975), esse fator atribui uma \emph{hierarquia de posições gramaticais}, de modo que o quantificador que está na posição mais acima tende a escopar sobre aquele mais abaixo: \emph{tópico $>$ sujeito superficial e profundo simultaneamente $>$ sujeito superficial ou profundo $>$ objeto preposicionado $>$ objeto indireto $>$ objeto direto}.
    \item Ordem linear ou superficial: \\
    Esse fator, que, segundo Filik et al. (2004), remete aos trabalhos de Johnson-Laird (1969) e Lakoff (1971) argumenta que o quantificador mais à esquerda na ordem linear da sentença tende a escopar sobre o elemento que está mais à direita.
    \item C-comando ou posição estrutural: \\
    Esse fator diz que o elemento que c-comanda tende a tomar escopo sobre o elemento c-comandado. Refinando uma proposta de Reinhart (1976), VanLehn (1978) propõe uma hierarquia de c-comando (\emph{apud} Tunstall, 1998: 32).
\end{enumerate}

Além dessas, alguns autores (Kurtzman \& MacDonald, 1993) tratam também de uma \emph{hierarquia temática}, em que o papel temático do NP quantificado poderia ter influência na tendência a tomar escopo amplo ou não. A hierarquia seria: \emph{agente $>$ experenciador $>$ tema}. Os resultados quanto a essa influência, no entanto, parecem inconclusivos (AUTORES???). Propõem-se, também, fatores não estruturais, como papel discursivo e conhecimento de mundo, que não serão abordados neste trabalho. Cabe-nos lembrar, no entanto, já que tem correlatos estruturais na teoria linguística, a posição de Fodor (1982), que propõe que \emph{tópico} tem preferência a escopar sobre \emph{comentário}.

\section{Estruturas sintáticas}
Além do comentário sobre os fatores que supostamente influenciariam a atribuição de escopo, é preciso destacar as estruturas sintáticas utilizadas pelos diversos trabalhos revisados. São elas:

\begin{enumerate}
    \item Sentenças ativas do tipo NP V NP:\\
    Sentenças como \emph{Every kid climbed a tree} ou \emph{A kid climbed every tree} são amplamente usadas nos estudos (Kurtzman \& MacDonald, 1993; Anderson, 2004; Dwived et al., 2010).
    \item Sentenças ativas do tipo NP V NP (Pred. Sec.):\\
    Sentenças como \emph{A boy sliced every carrot thin}, usadas por Tunstall (1998)~\cite{Tunstall1998} com propósito de diferenciar as propriedades lexicais de \emph{each} e \emph{every}.
    \item Sentenças com NPs complexos do tipo SUJ. V [NP$_1$ [NP$_2$]]:\\
    Esse tipo de sentença é usada por Kurtzman \& MacDonald (1993)~\cite{KMac1993}: \emph{George has [every photo [of an admiral]]}. Nelas, o segundo NP é complemento do primeiro. Na literatura linguística, são chamadas de estruturas de \emph{inverse link} (May \& Bale, 2007; Tunstall, 1998: 86).
    \item Sentenças passivas do tipo NP V$_{Aux}$ V$_{Princ.}$ by NP:\\
    Sentenças como \emph{Every tree was climbed by a kid}. Também usadas por Kurtzman \& MacDonald (1993)~\cite{KMac1993} sem resultados conclusivos.
    \item Sentenças dativas do tipo SUJ. V OD prep. OI: \\
    Sentenças como \emph{Kelly showed a photo to every critic} são usadas por Tunstall (1998)~\cite{Tunstall1998}, Filik et al. (2004) e Paterson et al. (2006). A escolha desse tipo de sentença, e das de duplo objeto, remonta a Micham et al. (1980) e Gillen (1991).
    \item Sentenças de duplo objeto do tipo SUJ. V OI OD: \\
    Com a inversão da posição linear do objeto indireto e a queda da preposição, tem-se sentenças como \emph{Kelly showed every critic a photo}. Esse uso é feito pelos mesmos autores citados no tópico anterior.
\end{enumerate}

Alguns desses tipos podem ser logo excluídos. As sentenças com NPs complexos têm uma forte tendência a realizar escopo inverso devido a motivos estruturais. As sentenças ativas com predicados secundários são usados para objetivos específicos por Tunstall (1998)~\cite{Tunstall1998}. As sentenças passivas, no entanto, parecem demonstram com menor intensidade uma preferência de escopo (Kurtzman \& MacDonald, 1993) não obtêm preferência de leitura com elas; Catlin \& Micham (1975) e Gillen (1991) (\emph{apud} Tunstall, 1998: 81) também não encontram preferência. Sobram, então, as sentenças ativas simples, as dativas e as de duplo objeto.

Argumentaremos, com Micham et al. (1980), Gillen (1991) e Tunstall (1998)~\cite{Tunstall1998} que as ativas simples não são boas para investigação de escopo de quantificadores, visto que muitos dos fatores que possivelmente influenciariam escopo dão preferência de escopo ao NP na posição de sujeito sobre o NP na posição de objeto (é \emph{sujeito superficial} e o \emph{argumento externo}, está na primeira posição na \emph{ordem linear}, em uma posição de \emph{c-comando} e, possivelmente, está em posição de \emph{tópico discursivo} enquanto objeto está na posição de \emph{comentário}).

As sentenças dativas e de duplo objeto, no entanto, reduzem essa forte tendência. Por exemplo, nas sentenças dativas, a posição gramatical favorece escopo do objeto indireto sobre o direto (OI $>$ OD), segundo a hierarquia de Ioup (1975). A ordem linear, no entanto, ``compensa'' esse direcionamento, visto que favorece escopo do objeto direto sobre o indireto (OD $>$ OI). Esse ponto é claramente apresentado por Paterson et al. (2006)~\cite{paterson2007}.

\section{Tipos de tarefas realizadas}
Outro aspecto importante a se destacar quanto aos trabalhos estudados diz respeito ao tipo de tarefa experimental utilizada para obter os dados e justificar as análises. Basicamente, há dois grandes grupos: as tarefas \emph{off-line} e as tarefas \emph{on-line}. No primeiro grupo estão os questionários impressos realizados por Tunstall (1998)~\cite{Tunstall1998} e Anderson (2004)~\cite{Anderson2004}, mas também as perguntas de compreensão acopladas a tarefas \emph{on-line}, como as de Kurtzman \& MacDonald (1993)~\cite{KMac1993} e de Anderson (2004)~\cite{Anderson2004}. Esses resultados não refletem o processamento imediato, mas uma análise posterior, possivelmente influenciada por fatores outros, como conhecimento de mundo. Assim, cabe destacar que, apesar de realizarem tarefas \emph{on-line}, Kurtzman \& MacDonald (1993)~\cite{KMac1993} e Anderson (2004)~\cite{Anderson2004}, via de regra, não obtiveram resultados robustos advindos dessas tarefas, mas apenas dos julgamentos \emph{off-line}.

No segundo grupo, por sua vez, estão as tarefas que mensuram com mais ou menos acurácia o processamento em tempo real, independente da reflexão consciente a respeito das sentenças lidas. Os resultados de leitura automonitorada (\emph{self-paced reading}) de Tunstall (1998)~\cite{Tunstall1998} e Anderson (2004)~\cite{Anderson2004} estão nesse grupo. Mas há também métodos mais refinados, como o uso de rastreamento ocular (\emph{eye-tracking}) (Filik et al., 2004; Paterson et al., 2006) e os potenciais evocados por eventos (\emph{event-related brain potentials - ERPs}) utilizados por Dwivedi et al. (2010). Com o uso dessas técnicas tem-se uma leitura mais precisa quanto ao funcionamento do processador humano.

Quanto às tarefas \emph{off-line}, cabe destacar os diversos tipos de perguntas utilizadas. Kurtzman \& MacDonald (1993)~\cite{KMac1993} se valem de uma pergunta relacionada à sentença continuativa que questionava se ela ``fazia sentido'' (\emph{made sense}) e era uma ``continuação natural'' (\emph{natural continuation}) da sentença quantificada. Tunstall (1998)~\cite{Tunstall1998} valia-se de uma tarefa de ``parar de fazer sentido'' (\emph{stop-making sense}), na qual os participantes interrompiam a leitura automonitorada assim que achavam que a frase parava de fazer sentido. Anderson (2004)~\cite{Anderson2004}, apesar de usar uma escala linear de 1 a 5 em um dos experimentos, em que 1 mostrava uma continuação singular e 5 uma continuação plural e o participante deveria marcar qual a mais plausível, em geral valeu-se de uma pergunta do tipo ``quantos'' (\emph{how many}), a que os participantes poderiam responder com ``um'' (\emph{one}) ou ``vários'' (\emph{several}). Filik et al. (2004)~\cite{filik2004} e Paterson et al. (2006)~\cite{paterson2007} investigam suas sentenças quantificadas previamente aos experimentos em si, apresentando-as a participantes que as julgavam em uma escala de 1 a 5 como se referindo ``definitivamente a uma entidade'' (\emph{definitely one}) ou ``definitivamente a mais de uma entendidade'' (\emph{definitely more then one}). No caso do estudo com \emph{each}, os autores encontram uma diferença entre as ordem \emph{a-each} e \emph{each-a} nas sentenças de duplo objeto, mas não encontram com as dativas, argumentando, portanto, que não há viés para uma ou outra leitura com essas sentenças (Paterson et al., 2006: 6).

A tabela~\ref{tab:t1} apresenta um resumo dos resultados obtidos quanto à preferência de interpretação dado o tipo de tarefa experimental realizado (uma revisão completa dos experimentos será feita no trabalho final, estando além do escopo deste projeto).

\begin{table}[h]
    \centering
    \begin{tabular}{lll}
    \toprule
         & a-every & every-a  \\ [.3em]
         \midrule
         off-line & 7(8) & 7(7) \\
         on-line & 4(6) & 2(7) \\
    \bottomrule
    \end{tabular}
    \caption{Preferência por escopo linear nos experimentos revisados}
    \label{tab:t1}
\end{table}

Conforme a tabela~\ref{tab:t1}, os estudos parecem indicar que:

\begin{enumerate}[label=(\roman*)]
    \item a preferência por escopo linear ocorre tanto para ordem \emph{a-every} quando para a ordem \emph{every-a} quando a tarefa é \emph{off-line}, quando o que se está mensurando demanda uma reflexão consciente, possivelmente metalinguística, sobre a sentença lida;
    \item a preferência por escopo linear, todavia, não ocorre ou ocorre em menor escala quando a tarefa é \emph{on-line}, ou seja, quando o que se está mensurando não demanda reflexão consciente, dependendo exclusivamente do processamento em tempo real.
\end{enumerate}

Esses dados parecem estar de acordo com os apontamentos de Rodrigues \& Marcilese (2014) sobre o português listados ao final do capítulo anterior.

\section{Competição ou subespecificação}
O terceiro ponto a se destacar nos estudos é o embate entre a ideia de \emph{competição de representações} e \emph{subespecificação}, que discutiremos aqui estritamente vinculado aos resultados obtidos com as diferentes ordens lineares dos quantificadores (\emph{a-every} e \emph{every-a}).

Em geral, os resultados mostram uma preferência por escopo linear (\emph{continuação singular}) quando a ordem dos quantificadores é \emph{a-every} (Kurtzman \& MacDonald, 1993; Tunstall, 1998; Anderson, 2004, Filik et al., 2004~\cite{filik2004}). Quando a ordem é \emph{every-a}, no entanto, os resultados divergem. Kurtzman \& MacDonald (1993)~\cite{KMac1993} encontram uma tendência para escopo linear (\emph{continuação plural}), mas dizem que é uma tendência menos robusta do que para \emph{a-every}. Tunstall (1998)~\cite{Tunstall1998} e Dwivedi et al. (2010)~\cite{Dwivedi2010} não encontram essa preferência. Apenas Anderson (2004)~\cite{Anderson2004} tem resultados que parecem indicar uma preferência pelo escopo linear, seja com contexto prévio (experimento 3) seja em frases isoladas (experimento 4). Ver também os resultados do experimento 5.

A fim de explicar os resultados mais robustos com a ordem \emph{a-every}, Kurtzman \& MacDonald (1993: 257)~\cite{KMac1993}, seguindo proposta de Fodor (1982), propõem o princípio de referência unitária (\emph{single reference principle}), segundo o qual o processador, ao encontrar o artigo indefinido \emph{a} na posição de sujeito, imediatamente atribui a ele referência a uma única entidade. Assim, apenas quando encontra o quantificador universal \emph{every} na posição de objeto, é que surge a possibilidade de mudar de referência. Esse processo de mudança, no entanto, é custoso, de modo que o processador evitará fazê-lo, explicando a maior tendência de escopo linear (\emph{a $>$ every}). Filik et al. (2004)~\cite{filik2004} testam essa proposição com uso de técnica de rastreamento ocular. Obtendo maiores tempos para a ordem linear \emph{a-every} do que para \emph{every-a}, argumentam que têm indício de que de fato o que ocorre é um processo de revisão da representação até então construída.

Tal explicação, contudo, não é unânime. Tunstall (1988: 70)~\cite{Tunstall1998}, ao não obter preferência de escopo com a ordem linear \emph{every-a}, propõe o princípio da vagueza (\emph{vagueness principle}), segundo o qual o processador, ao se deparar com \emph{every} na primeira posição, mantém a sua referência vaga (ou subespecificada) quanto ao número de elementos envolvidos, que deverá ser definida de acordo com o contexto subsequente. Já que a representação é deixada em aberto, tanto a continuação plural (escopo linear) quanto a continuação singular (escopo inverso) são possíveis, de modo que não há preferência por nenhuma delas. Dwivedi et al. (2010)~\cite{Dwivedi2010} também defendem esse ponto, visto que seus resultados de potenciais evocados (ERPs) não apresentam N400 ou P600, que indicariam revisão das representações previamente estabelecidas. Em vez de uma competição entre representações, propõe-se uma subespecificação de representações.

\chapter{Objetivos}
Dada a revisão da literatura apresentada anteriormente, alguns objetivos se colocam neste trabalho, sendo o principal deles \emph{investigar o comportamento do processador humano em tempo real durante a leitura de sentenças duplamente quantificadas ambíguas ou potencialmente ambíguas}. Uma vez que a literatura apresenta hipóteses distintas sobre o comportamento do \emph{parser}, a fim de alcançar o objetivo apresentado anteriormente, este projeto tem por meta, também, prover evidência que auxilie na seleção das hipóteses em competição. Ao longo desse percurso, almeja-se angariar evidências que ajudem, ainda, em uma melhor descrição de determinadas estruturas linguísticas do português brasileiro, bem como na descrição do comportamento de \emph{todo+NP}, expressão que parece gerar grande debate, como visto na revisão da literatura.

Uma revisão mais aprofundada da literatura sobre as técnicas experimentais utilizadas no estudo de escopo de quantificadores, bem como os impactos que elas têm sobre os resultados experimentais obtidos também se enquadra nos objetivos deste trabalho. Com isso, será possível contribuir para o debate teórico e metodológico da área, uma vez que muitos dos resultados conflitantes mostrados na literatura estão vinculados ao uso de diferentes técnicas e diferentes estruturas linguísticas que não são, muitas vezes, comparáveis diretamente. Fornecer evidências que apontem para uma métrica que permita vislumbrar a comensurabilidade do fenômeno em questão também se enquadra nos objetivos deste trabalho, já que ajudam a pensar sobre como investigar o comportamento do \emph{parser}.\\
\\
As perguntas centrais que embasam a pesquisa são:

\begin{enumerate}
  \item Como o processador humano ou \emph{parser} processa sentenças com os quantificadores universal \emph{todo} e indefinido \emph{um}?
  \item O processador humano, ao processar sentenças duplamente quantificadas, atua de modo algorítimico ou se utiliza de heurísticas?
  \item O processador humano se vale de uma leitura preferencial ou \emph{default} ou mantém as leituras subespecificadas até que informação de natureza estrutural ou contextual lhe permita selecionar uma delas?
\end{enumerate}

\noindent A fim de responder tais perguntas, deve-se responder também:

\begin{enumerate}[label=(\alph*)]
  \item Quais as características estruturais e semântico-pragmáticas de sentenças com \emph{todo} e \emph{um} são relevantes a fim de se construir sentenças verdadeiramente ambíguas no que diz respeito a escopo de quantificadores?
  \item Como evitar sentenças enviesadas (semantica ou pragmanticamente), de modo que o comportamento do \emph{parser} possa ser investigado em estado natural?
  \item Quais técnicas experimentais discutidas na literatura permitiriam investigar o fenômeno em estudo de modo preciso e rigoroso?
  \item Como conciliar os resultados muitas vezes conflitantes da literatura a partir dos resultados a serem obtidos?
\end{enumerate}

\chapter{Metodologia}
Este trabalho pretende valer-se exclusivamente de metodologia experimental, visto vincular-se ao modelo de pesquisa rotineiramente desenvolvido no LAPAL-PUC-Rio e ser o método que permite um controle preciso e rigoroso dos inúmeros fatores que governam o fenômeno em questão. Já que fatores semânticos e pragmáticos podem interferir e até mesmo determinar certas leituras, investigar o comportamento do \emph{processador humano} depende de um refinado controle das potenciais inteferências, linguísticas ou não, que possam estar presentes no caso em questão. A metodologia experimental é aquela que melhor se adéqua, portanto, a esse tipo de estudo.

Ademais, como informado na revisão da literatura, quando se trata de técnica \emph{off-line}, o escopo de quantificadores parece convergir, na grande maioria dos casos, em uma leitura preferencial das sentenças estudadas. Entretanto, quando se trata de técnica \emph{on-line}, os resultados são conflitantes. Sendo assim, o uso de ambas as metodologias mostram-se fundamentais para o estudo do comportamento de \emph{todo+NP} no português brasileiro.

Sendo assim, o primeiro experimento que pretendemos realizar será uma tarefa de leitura automonitorada (\emph{self-paced reading}) com julgamento de gramaticalidade usando a \emph{escala Likert} (REFERÊNCIAS). Este estudo teria caráter normativo, visando estabelecer um conjunto mínimo de sentenças que demonstrassem \textcolor{red}{``ativar''} comportamentos semelhantes nos falantes e fossem efetivamente ambíguas quanto às leituras distributiva e coletiva. A partir da discussão sobre as propriedades linguísticas de \emph{todo} no português brasileiro, pode-se dizer que deveremos elaborar sentenças semelhantes àquelas abaixo:

\begin{enumerate}
  \item O vendedor mostrou toda pulseira de ouro para um comprador.
  \item O vendedor mostrou uma pulseira para todo comprador de joias.
\end{enumerate}

Essa tarefa deverá ser realizada exclusivamente em ambiente virtual, com o uso da ferramenta de elaboração de experimentos disponível \emph{online} chamada \emph{Ibex Farm}, acessível em: \emph{https://spellout.net/ibexfarm/}. O comportamento dos falantes ao julgar tais sentenças permitirá a elaboração de um conjunto de frases normatizadas que se comportam de modo semlhante, garantindo maior rigor na construção do segundo experimento.

Nessa segunda atividade, pretendemos realizar uma tarefa de leitura automonitorada seguida de uma frase continuativa com um nome anafórico que retome o substantivo encabeçado pelo quantificador indefinido \emph{um}. Tais estímulos deverão se assemelhar àqueles abaixo:

\begin{enumerate}[resume]
  \item O vendedor mostrou toda pulseira de ouro para um comprador,\\ mas \textbf{o homem/os homens} não era/eram muito decidido/decididos.
  \item O vendedor mostrou uma pulseira para todo comprador de joias,\\ mas \textbf{a peça/as peças} não era/eram muito bonita/bonitas.
\end{enumerate}

Essa atividade permitirá a medição do tempo de reação do participante nas posições do nome anafórico, do verbo que se segue e do advérbio seguinte, de modo que será possível verificar a reação imediata, espontânea e não mediada por integração de informação contextual e/ou semantico-pragmática, permitindo visualizar a leitura que o \emph{parser} realiza da sentença. Assim, será possível visumbrar, também, se há ou não uma leitura preferencial ou se a especificação fica vaga.

Essa tarefa deverá ser realizada em ambiente virtual, usando, também, o \emph{Ibex Farm}. No entanto, como ela envolve a medição do tempo de reação dos participantes, seria interessante que, caso as circunstâncias permitam, ela seja complementada com uma tarefa presencial, no laboratório, onde não há potenciais distrações externas e o experimento está mais prontamente sob controle do pesquisador.

Em paralelo a tais experimentos, pretendemos, também, realizar um experimento de leitura automonitorada com \emph{twist}, técnica desenvolvida por Patson \& Warren (2010)~\cite{PatWarren2010} e usada por Dwivedi \& Gibson (2017)~\cite{DwiGib2017}. Apesar de a técnica do experimento apresentado anteriormente mensurar o comportamento do \emph{parser} em tempo real, ela o faz na sentença continuativa, não na sentença quantificada. Essa técnica, no entanto, permite investigar se o escopo é estabelecido \emph{in loco}, ou seja, na própria sentença quantificada, assim que o participante se depara com o segundo quantificador da sentença. As frases utilizadas serão as mesmas normatizadas pelo experimento \emph{off-line} já discutido.

Como as tarefas anteriores, essa também se valerá do \emph{Ibex Farm}, e, já que envolve controle de tempo, seria ideal que pudesse ser realizada no laboratório, onde as condições são mais controladas.

A depender, também, das circunstâncias sanitárias do país, pretendemos realizar uma tarefa de rastreamento ocular (\emph{eye tracking}), essa exclusivamente em laboratório, que possui recursos tecnológicos próprios para tal, vinculada ao \emph{paradigma do mundo visual} (REFERÊNCIAS). Com esse tipo de técnica seria possível mensurar a movimentação do olhar dos participantes e perceber a leitura que fazem das sentenças em tempo real, antes mesmo que ele possa realizar qualquer reflexão metalinguística sobre o fenômeno. A construção das sentenças seguiria os mesmos padrões das tarefas anteriores e dependeria, obviamente, dos resultados obtidos naqueles experimentos.

Por fim, a depender dos resultados obtidos e das circunstâncias, pretendemos realizar um experimento de produção, de caráter exploratório, em que o participante, após ler/ouvir a sentença quantificada, deveria dar uma continuação a ela a partir de alguma pista. Essa tarefa é importante porque permite verificar a interpretação de fato construída pelo falante e não a sua reação a interpretações já dadas, caso de todos os experimmentos anteriores.

Em todas as tarefas, os riscos de participação são mínimos. As sentenças serão construídas de modo a evitar quaisquer expressões ou descrever quaisquer cenas que possam por ventura causar constrangimentos aos participantes. Entretanto, ainda assim é possível que o participante sinta leve desconforto por se manter sentado(a) e parcialmente imóvel durante a realização das tarefas. Esse desconforto será tentativamente minimizado ao construirmos experimentos com poucos estímulos experimentais (aproximadamente 50 sentenças) e, quando feitos no laboratório, disponibilizaremos um local tranquilo e confortável com um pequeno intervalo na metade da atividade se o participante desejar.

O participante também será informado que se encontra livre para encerrar a atividade a qualquer momento sem qualquer prejuízo ou constrangimento à sua pessoa.

Especificamente sobre o uso de rastreador ocular (\emph{eye tracker}, os riscos são mínimos, similares aos de atividades diárias como uso de computador e de televisão. A luz infravermelha invisível emitida pelo aparelho assemelha-se a luzes naturais e artificiais (como o sol, o fogo, velas e certas lâmpadas) presentes em vários ambientes. O aparelho é testado de acordo com as normas europeias de segurança, de forma a ser considerado inofensivo aos seres humanos.

Por fim, garantiremos ao participante -- e ele será informado a respeito antes do início da atividade -- que suas informações serão tratadas com o mais absoluto sigilo e confidencialidade, de modo a preservar a sua identidade. Os resultados da pesquisa serão divulgados em eventos e publicações científicas, para fins acadêmicos apenas, sendo mantido o anonimato dos participantes.

\input{Cap7_Cronograma.tex}

\postextual
\newpage
\chapter{Anexos}
Apresentamos, abaixo, o textos dos Termos de Consentimento Livre e Esclarecido (TCLE) a serem fornecidos aos participantes antes da realização da tarefa a fim de garantir que estejam bem informados quanto à proposta da pesquisa e seus riscos potenciais. Todos os termos impressos serão encabeçados pela logomarca da PUC-Rio, pela expressão ``Termo de Consentimento Livre e Esclarecido'' e pelo nome da tarefa a ser realizada (\emph{v.g.} ``Atividade de leitura com captura de dados oculares'').

\section{Atividade de leitura e julgamento \emph{Online}}
O termo de consentimento abaixo será usado para a tarefa de julgamento de gramaticalidade com \emph{escala Likert}.

\begin{center}
  \textbf{Termo de Consentimento Livre e Esclarecido}
\end{center}
Convidamos você a participar como voluntário(a) da pesquisa intitulada ``Estruturas distributivas no Português Brasileiro: aspectos linguísticos e processuais'', de responsabilidade do doutorando Igor de Oliveira Costa, orientando da profª. Erica dos Santos Rodrigues, do Programa de Pós-Graduação Estudos da Linguagem (PPGEL), PUC-Rio.
\\
\\
OBJETIVOS: O objetivo da pesquisa é investigar como os falantes do Português Brasileiro interpretam expressões linguísticas usadas para especificar a quantos elementos a frase está remetendo.
\\
\\
JUSTIFICATIVA: O estudo irá contribuir (i) para o desenvolvimento de pesquisas em Linguística Teórica sobre as estruturas de quantificação no Português Brasileiro e (ii) para a avaliação, em Psicolinguística, de como as pessoas compreendem as referidas estruturas em situações controladas.
\\
\\
PROCEDIMENTOS: Nesta tarefa você irá ler um conjunto de sentenças em português e informará a sua interpretação sobre elas. Não se trata de um teste de gramática, com respostas certas ou erradas. A proposta é registrar sua resposta mais natural e espontânea sobre as estruturas da língua. A realização da tarefa dura alguns minutos apenas.
\\
\\
Antes de realizar a tarefa, você preencherá uma ficha com alguns dados pessoais, como idade, escolaridade e gênero/sexo. Seu nome não será registrado e em hipótese alguma faremos referência à sua identidade.
\\
\\
RISCOS: É possível que você sinta leve desconforto e cansaço por se manter sentado(a) e concentrado na leitura das frases apresentadas. Entretanto, a atividade é de curta duração, sendo mínimos os riscos envolvidos na realização da tarefa, similares aos de atividades diárias com o uso de computador.
\\
\\
BENEFÍCIOS: Você não pagará e nem será remunerado(a) por sua participação. Sua participação voluntária irá, contudo, contribuir para as pesquisas, tanto em Linguística Teórica quanto em Psicolinguística, sobre estruturas sintáticas envolvendo relações de quantificação.
\\
\\
DIVULGAÇÃO E CONFIDENCIALIDADE: Esclarecemos que sua participação é totalmente voluntária, podendo: recusar-se a participar, ou mesmo desistir a qualquer momento, sem que isto acarrete qualquer ônus ou prejuízo a sua pessoa. Esclarecemos, também, que suas informações serão tratadas com o mais absoluto sigilo e confidencialidade, de modo a preservar a sua identidade. Os resultados da pesquisa serão divulgados em eventos e publicações científicas, sendo mantido o anonimato dos participantes.
\\
\\
INFORMAÇÕES ADICIONAIS: Contatos para esclarecimentos de dúvidas sobre a pesquisa e seus aspectos éticos:
\\
\\
Câmara de Ética em Pesquisa da PUC-Rio (CEPq-PUC-Rio), situado  na Rua Marquês de São Vicente, 225 -- Edifício Kenedy, 2º andar. Gávea -- Rio de Janeiro/RJ, CEP: 22453-900; Telefone: + 55 (21) 3527-1618.
\\
\\
Programa de Pós-Graduação Estudos da Linguagem -- Departamento de Letras da PUC--Rio, situado na Rua Marquês de São Vicente, 225 -- Edifício Pe. Leonel Franca, 3º andar. Gávea -- Rio de Janeiro/RJ, CEP: 22451-900; Telefone: +55 (21) 3527-1297
\\
\\
Caso deseje esclarecimentos adicionais sobre a pesquisa, os pesquisadores podem ser contactados nos seguintes e-mails:
\\
\\
E-mail do doutorando: igordeo.costa@gmail.com \\
E-mail da orientadora: ericasr@puc-rio.br
\\
\\
Este termo de consentimento será arquivado pelos pesquisadores responsáveis e uma cópia será enviada a você por e-mail. Os dados coletados na pesquisa ficarão arquivados com o pesquisador responsável por um período de 5 (cinco) anos. Decorrido este tempo, os pesquisadores avaliarão os documentos para a sua destinação final, de acordo com a legislação vigente. Os pesquisadores tratarão a sua identidade com padrões profissionais de sigilo, atendendo a legislação brasileira (Resoluções Nº 510/16 e Nº 466/12 do Conselho Nacional de Saúde), utilizando as informações somente para os fins acadêmicos e científicos.
\\
\\
Declaro que li com atenção o termo de consentimento livre e esclarecido da pesquisa “Estruturas distributivas no Português Brasileiro: aspectos linguísticos e processuais” e acredito estar suficientemente informado sobre o estudo, ficando claro que minha participação é voluntária e que posso retirar este consentimento a qualquer momento sem penalidades.
\\
\\
Ao clicar nessa opção, expresso minha concordância em participar deste estudo.
\\
\\
ACEITO PARTICIPAR

\section{Atividades de leitura automonitorada \emph{Online} e presencial}
O termo que se segue será usado para os experimentos de leitura automonitorada e leitura automonitorada com \emph{twist}.

\begin{center}
  \textbf{Termo de Consentimento Livre e Esclarecido}
\end{center}
Convidamos você a participar como voluntário(a) da pesquisa intitulada ``Estruturas distributivas no Português Brasileiro: aspectos linguísticos e processuais'', de responsabilidade do doutorando Igor de Oliveira Costa, orientando da profª. Erica dos Santos Rodrigues, do Programa de Pós-Graduação Estudos da Linguagem (PPGEL), PUC-Rio.
\\
\\
OBJETIVOS: O objetivo da pesquisa é investigar como os falantes do Português Brasileiro interpretam expressões linguísticas usadas para especificar a quantos elementos a frase está remetendo.
\\
\\
JUSTIFICATIVA: O estudo irá contribuir (i) para o desenvolvimento de pesquisas em Linguística Teórica sobre as estruturas de quantificação no Português Brasileiro e (ii) para a avaliação, em Psicolinguística, de como as pessoas compreendem as referidas estruturas em situações controladas.
\\
\\
PROCEDIMENTOS: Nesta tarefa você irá ler um conjunto de sentenças em português e \textcolor{blue}{responderá, apertando teclas do computador, algumas perguntas de compreensão sobre elas}. Não se trata de um teste de gramática, com respostas certas ou erradas. A proposta é registrar sua resposta mais natural e espontânea sobre as estruturas da língua. A realização da tarefa dura alguns minutos apenas. \textcolor{blue}{Durante a realização da tarefa, o computador registrará o tempo que você leva para passar de uma palavra a outra.}
\\
\\
Antes de realizar a tarefa, você preencherá uma ficha com alguns dados pessoais, como idade, escolaridade e gênero/sexo. Seu nome não será registrado e em hipótese alguma faremos referência à sua identidade.
\\
\\
RISCOS: É possível que você sinta leve desconforto e cansaço por se manter sentado(a) e concentrado na leitura das frases apresentadas. Entretanto, a atividade é de curta duração, sendo mínimos os riscos envolvidos na realização da tarefa, similares aos de atividades diárias com o uso de computador.
\\
\\
BENEFÍCIOS: Você não pagará e nem será remunerado(a) por sua participação. Sua participação voluntária irá, contudo, contribuir para as pesquisas, tanto em Linguística Teórica quanto em Psicolinguística, sobre estruturas sintáticas envolvendo relações de quantificação.
\\
\\
DIVULGAÇÃO E CONFIDENCIALIDADE: Esclarecemos que sua participação é totalmente voluntária, podendo: recusar-se a participar, ou mesmo desistir a qualquer momento, sem que isto acarrete qualquer ônus ou prejuízo a sua pessoa. Esclarecemos, também, que suas informações serão tratadas com o mais absoluto sigilo e confidencialidade, de modo a preservar a sua identidade. Os resultados da pesquisa serão divulgados em eventos e publicações científicas, sendo mantido o anonimato dos participantes.
\\
\\
INFORMAÇÕES ADICIONAIS: Contatos para esclarecimentos de dúvidas sobre a pesquisa e seus aspectos éticos:
\\
\\
Câmara de Ética em Pesquisa da PUC-Rio (CEPq-PUC-Rio), situado  na Rua Marquês de São Vicente, 225 -- Edifício Kenedy, 2º andar. Gávea -- Rio de Janeiro/RJ, CEP: 22453-900; Telefone: + 55 (21) 3527-1618.
\\
\\
Programa de Pós-Graduação Estudos da Linguagem -- Departamento de Letras da PUC--Rio, situado na Rua Marquês de São Vicente, 225 -- Edifício Pe. Leonel Franca, 3º andar. Gávea -- Rio de Janeiro/RJ, CEP: 22451-900; Telefone: +55 (21) 3527-1297
\\
\\
Caso deseje esclarecimentos adicionais sobre a pesquisa, os pesquisadores podem ser contactados nos seguintes e-mails:
\\
\\
E-mail do doutorando: igordeo.costa@gmail.com \\
E-mail da orientadora: ericasr@puc-rio.br
\\
\\
Este termo de consentimento será arquivado pelos pesquisadores responsáveis e uma cópia será enviada a você por e-mail. Os dados coletados na pesquisa ficarão arquivados com o pesquisador responsável por um período de 5 (cinco) anos. Decorrido este tempo, os pesquisadores avaliarão os documentos para a sua destinação final, de acordo com a legislação vigente. Os pesquisadores tratarão a sua identidade com padrões profissionais de sigilo, atendendo a legislação brasileira (Resoluções Nº 510/16 e Nº 466/12 do Conselho Nacional de Saúde), utilizando as informações somente para os fins acadêmicos e científicos.
\\
\\
\noindent \textbf{Para o caso de atividade \emph{online}, será adicionado o seguinte trecho:}\\
Declaro que li com atenção o termo de consentimento livre e esclarecido da pesquisa ``Estruturas distributivas no Português Brasileiro: aspectos linguísticos e processuais'' e acredito estar suficientemente informado sobre o estudo, ficando claro que minha participação é voluntária e que posso retirar este consentimento a qualquer momento sem penalidades.
\\
\\
Ao clicar nessa opção, expresso minha concordância em participar deste estudo.
\\
\\
ACEITO PARTICIPAR
\\
\noindent \textbf{Para o caso de atividade presencial, será adicionado o seguinte trecho:}\\
\\
\noindent CONSENTIMENTO\\
Eu, \rule[-1mm]{8cm}{0.3mm}, CPF \rule[-1mm]{5cm}{0.3mm}, após a leitura deste documento e de ter tido a oportunidade de esclarecer todas as minhas dúvidas, acredito estar suficientemente informado(a), ficando claro que minha participação no projeto ``Estruturas distributivas no Português Brasileiro: aspectos linguísticos e processuais'' é voluntária e que posso retirar este consentimento a qualquer momento sem penalidades ou perda de qualquer benefício. Estou ciente também dos objetivos da pesquisa, dos procedimentos aos quais serei submetido(a), dos possíveis danos ou riscos deles provenientes e da garantia de confidencialidade e esclarecimentos sempre que desejar.
Diante do exposto expresso minha concordância de espontânea vontade em participar deste estudo.

\begin{center}
  Rio de Janeiro, \rule[-1mm]{1cm}{0.3mm} de \rule[-1mm]{3cm}{0.3mm} de \rule[-1mm]{1.5cm}{0.3mm}.\\
  \vspace{1cm}
  \rule[-1mm]{8cm}{0.3mm}\\
  (Assinatura do participante da pesquisa)\\
  \vspace{1cm}
  \rule[-1mm]{8cm}{0.3mm}\\
  (Igor de Oliveira Costa -- pesquisador responsável)
\end{center}

\section{Atividade de rastreamento ocular}
O termo de consentimento abaixo será usado para a tarefa de rastreamento ocular a ser realizada presencialmente nas instalações físicas do LAPAL/PUC-Rio ou em outras dependências da PUC-Rio e/ou de outras universidades.

\begin{center}
  \textbf{Termo de Consentimento Livre e Esclarecido}
\end{center}
Você está sendo convidado(a) a participar da Pesquisa ``Estruturas distributivas no Português Brasileiro: aspectos linguísticos e processuais'', de responsabilidade do doutorando Igor de Oliveira Costa, orientando da profa. Erica dos Santos Rodrigues, do Programa de Pós-Graduação Estudos da Linguagem (PPGEL), PUC-Rio. Antes de aceitar participar desta pesquisa, é necessária sua compreensão a respeito das informações e das instruções contidas neste documento. Os pesquisadores responderão todas as dúvidas antes que decida participar. A qualquer momento, poderá interromper sua participação na pesquisa sem sofrer qualquer penalização ou constrangimento.
\\
\\
OBJETIVOS: O objetivo da pesquisa é investigar como os falantes do Português Brasileiro interpretam expressões linguísticas usadas para especificar a quantos elementos a frase está remetendo. Na atividade experimental de que você irá participar serão capturados e registrados seus movimentos oculares durante a leitura de sentenças e/ou apresentação de imagens acompanhadas da escuta dessas frases, de modo a avaliar hipóteses sobre como as estruturas sintáticas testadas são compreendidas.
\\
\\
JUSTIFICATIVA: O estudo irá contribuir (i) para o desenvolvimento de pesquisas em Linguística Teórica sobre as estruturas de quantificação no Português Brasileiro e (ii) para a avaliação, em Psicolinguística, de como as pessoas compreendem as referidas estruturas em situações controladas.
\\
\\
PROCEDIMENTOS: Você irá ler ou ouvir um conjunto de sentenças em português, que podem estar acompanhadas ou não de imagens, com apresentação em computador, e irá responder a algumas perguntas simples de compreensão. Não se trata de um teste de gramática, com respostas certas ou erradas. A proposta é registrar sua resposta mais natural e espontânea sobre as estruturas da língua. Durante a realização da tarefa, seus movimentos oculares serão gravados com uso de um equipamento de rastreamento ocular. O equipamento detectará os movimentos oculares a partir de reflexos gerados na córnea por uma luz infravermelha emitida pelo equipamento, similar a luzes naturais e artificiais. Será necessário realizar, antes da apresentação dos estímulos, um procedimento de ``calibração'' do equipamento, durante o qual você será solicitado(a) a olhar para pontos apresentados na tela de computador. Você receberá orientação para se posicionar de maneira confortável na cadeira em que se encontra e será solicitado que evite realizar movimentos bruscos durante a sessão para preservar a qualidade da calibração realizada. A sessão experimental terá duração aproximada de 20 minutos. Ao final da tarefa, você será solicitado(a) a preencher um questionário sociodemográfico, com algumas perguntas sobre idade, sexo, escolaridade e história linguística.
\\
\\
DESCONFORTOS E RISCOS ESPERADOS: É possível que você sinta leve desconforto por se manter sentado(a) e parcialmente imóvel durante a sessão. No entanto, buscamos minimizar ao máximo esse desconforto, realizando a tarefa em local tranquilo e confortável e fazendo um pequeno intervalo na metade da atividade se desejar. Você encontra-se livre para encerrar a atividade a qualquer momento. Os riscos envolvidos na realização da tarefa são mínimos, similares aos de atividades diárias como uso de computador e de televisão. A luz infravermelha invisível emitida pelo aparelho assemelha-se a luzes naturais e artificiais (como o sol, o fogo, velas e certas lâmpadas) presentes em vários ambientes. O aparelho é testado de acordo com as normas europeias de segurança, de forma a ser considerado inofensivo aos seres humanos.
\\
\\
BENEFÍCIOS PARA OS PARTICIPANTES: Você não pagará e nem será remunerado(a) por sua participação. Sua participação voluntária irá, contudo, contribuir para as pesquisas, tanto em Linguística Teórica quanto em Psicolinguística, sobre estruturas sintáticas envolvendo relações de quantificação.
\\
\\
DIVULGAÇÃO E CONFIDENCIALIDADE: Esclarecemos que sua participação é totalmente voluntária, podendo: recusar-se a participar, ou mesmo desistir a qualquer momento, sem que isto acarrete qualquer ônus ou prejuízo a sua pessoa. Esclarecemos, também, que suas informações serão tratadas com o mais absoluto sigilo e confidencialidade, de modo a preservar a sua identidade. Os resultados da pesquisa serão divulgados em eventos e publicações científicas, sendo mantido o anonimato dos participantes.
\\
\\
INFORMAÇÕES ADICIONAIS: Contatos para esclarecimentos de dúvidas sobre a pesquisa e seus aspectos éticos:
\\
\\
Câmara de Ética em Pesquisa da PUC-Rio (CEPq-PUC-Rio), situado  na Rua Marquês de São Vicente, 225 -- Edifício Kenedy, 2º andar. Gávea -- Rio de Janeiro/RJ, CEP: 22453-900; Telefone: + 55 (21) 3527-1618.
\\
\\
Programa de Pós-Graduação Estudos da Linguagem -- Departamento de Letras da PUC--Rio, situado na Rua Marquês de São Vicente, 225 -- Edifício Pe. Leonel Franca, 3º andar. Gávea -- Rio de Janeiro/RJ, CEP: 22451-900; Telefone: +55 (21) 3527-1297
\\
\\
Caso deseje esclarecimentos adicionais sobre a pesquisa, os pesquisadores podem ser contactados nos seguintes e-mails:
\\
\\
E-mail do doutorando: igordeo.costa@gmail.com \\
E-mail da orientadora: ericasr@puc-rio.br
\\
\\
Este termo de consentimento será arquivado pelos pesquisadores responsáveis e uma cópia será enviada a você por e-mail. Os dados coletados na pesquisa ficarão arquivados com o pesquisador responsável por um período de 5 (cinco) anos. Decorrido este tempo, os pesquisadores avaliarão os documentos para a sua destinação final, de acordo com a legislação vigente. Os pesquisadores tratarão a sua identidade com padrões profissionais de sigilo, atendendo a legislação brasileira (Resoluções Nº 510/16 e Nº 466/12 do Conselho Nacional de Saúde), utilizando as informações somente para os fins acadêmicos e científicos.
\\
\\
\noindent CONSENTIMENTO \\
Eu, \rule[-1mm]{8cm}{0.3mm}, CPF \rule[-1mm]{5cm}{0.3mm}, após a leitura deste documento e de ter tido a oportunidade de esclarecer todas as minhas dúvidas, acredito estar suficientemente informado(a), ficando claro que minha participação no projeto ``Estruturas distributivas no Português Brasileiro: aspectos linguísticos e processuais'' é voluntária e que posso retirar este consentimento a qualquer momento sem penalidades ou perda de qualquer benefício. Estou ciente também dos objetivos da pesquisa, dos procedimentos aos quais serei submetido(a), dos possíveis danos ou riscos deles provenientes e da garantia de confidencialidade e esclarecimentos sempre que desejar.
Diante do exposto expresso minha concordância de espontânea vontade em participar deste estudo.

\begin{center}
  Rio de Janeiro, \rule[-1mm]{1cm}{0.3mm} de \rule[-1mm]{3cm}{0.3mm} de \rule[-1mm]{1.5cm}{0.3mm}.\\
  \vspace{1cm}
  \rule[-1mm]{8cm}{0.3mm}\\
  (Assinatura do participante da pesquisa)\\
  \vspace{1cm}
  \rule[-1mm]{8cm}{0.3mm}\\
  (Igor de Oliveira Costa -- pesquisador responsável)
\end{center}
\vspace{2cm}

\section{Atividade de produção eliciada}

\begin{center}
  \textbf{Termo de Consentimento Livre e Esclarecido}
\end{center}
Convidamos você a participar como voluntário(a) da pesquisa intitulada ``Estruturas distributivas no Português Brasileiro: aspectos linguísticos e processuais'', de responsabilidade do doutorando Igor de Oliveira Costa, orientando da profª. Erica dos Santos Rodrigues, do Programa de Pós-Graduação Estudos da Linguagem (PPGEL), PUC-Rio.
\\
\\
OBJETIVOS: O objetivo da pesquisa é investigar como os falantes do Português Brasileiro interpretam expressões linguísticas usadas para especificar a quantos elementos a frase está remetendo.
\\
\\
JUSTIFICATIVA: O estudo irá contribuir (i) para o desenvolvimento de pesquisas em Linguística Teórica sobre as estruturas de quantificação no Português Brasileiro e (ii) para a avaliação, em Psicolinguística, de como as pessoas compreendem as referidas estruturas em situações controladas.
\\
\\
PROCEDIMENTOS: Nesta tarefa você irá ler um conjunto de sentenças em português e \textcolor{blue}{produzirá, a partir de pistas visuais fonecidas na tela do computador, frases que permitam dar continuidade a tais sentenças}. Não se trata de um teste de gramática, com respostas certas ou erradas. A proposta é registrar sua resposta mais natural e espontânea sobre as estruturas da língua. A realização da tarefa dura alguns minutos apenas.
\\
\\
Antes de realizar a tarefa, você preencherá uma ficha com alguns dados pessoais, como idade, escolaridade e gênero/sexo. Seu nome não será registrado e em hipótese alguma faremos referência à sua identidade.
\\
\\
RISCOS: É possível que você sinta leve desconforto e cansaço por se manter sentado(a) e concentrado na leitura das frases apresentadas. Entretanto, a atividade é de curta duração, sendo mínimos os riscos envolvidos na realização da tarefa, similares aos de atividades diárias com o uso de computador.
\\
\\
BENEFÍCIOS: Você não pagará e nem será remunerado(a) por sua participação. Sua participação voluntária irá, contudo, contribuir para as pesquisas, tanto em Linguística Teórica quanto em Psicolinguística, sobre estruturas sintáticas envolvendo relações de quantificação.
\\
\\
DIVULGAÇÃO E CONFIDENCIALIDADE: Esclarecemos que sua participação é totalmente voluntária, podendo: recusar-se a participar, ou mesmo desistir a qualquer momento, sem que isto acarrete qualquer ônus ou prejuízo a sua pessoa. Esclarecemos, também, que suas informações serão tratadas com o mais absoluto sigilo e confidencialidade, de modo a preservar a sua identidade. Os resultados da pesquisa serão divulgados em eventos e publicações científicas, sendo mantido o anonimato dos participantes.
\\
\\
INFORMAÇÕES ADICIONAIS: Contatos para esclarecimentos de dúvidas sobre a pesquisa e seus aspectos éticos:
\\
\\
Câmara de Ética em Pesquisa da PUC-Rio (CEPq-PUC-Rio), situado  na Rua Marquês de São Vicente, 225 -- Edifício Kenedy, 2º andar. Gávea -- Rio de Janeiro/RJ, CEP: 22453-900; Telefone: + 55 (21) 3527-1618.
\\
\\
Programa de Pós-Graduação Estudos da Linguagem -- Departamento de Letras da PUC--Rio, situado na Rua Marquês de São Vicente, 225 -- Edifício Pe. Leonel Franca, 3º andar. Gávea -- Rio de Janeiro/RJ, CEP: 22451-900; Telefone: +55 (21) 3527-1297
\\
\\
Caso deseje esclarecimentos adicionais sobre a pesquisa, os pesquisadores podem ser contactados nos seguintes e-mails:
\\
\\
E-mail do doutorando: igordeo.costa@gmail.com \\
E-mail da orientadora: ericasr@puc-rio.br
\\
\\
Este termo de consentimento será arquivado pelos pesquisadores responsáveis e uma cópia será enviada a você por e-mail. Os dados coletados na pesquisa ficarão arquivados com o pesquisador responsável por um período de 5 (cinco) anos. Decorrido este tempo, os pesquisadores avaliarão os documentos para a sua destinação final, de acordo com a legislação vigente. Os pesquisadores tratarão a sua identidade com padrões profissionais de sigilo, atendendo a legislação brasileira (Resoluções Nº 510/16 e Nº 466/12 do Conselho Nacional de Saúde), utilizando as informações somente para os fins acadêmicos e científicos.
\\
\\
\noindent CONSENTIMENTO \\
Eu, \rule[-1mm]{8cm}{0.3mm}, CPF \rule[-1mm]{5cm}{0.3mm}, após a leitura deste documento e de ter tido a oportunidade de esclarecer todas as minhas dúvidas, acredito estar suficientemente informado(a), ficando claro que minha participação no projeto ``Estruturas distributivas no Português Brasileiro: aspectos linguísticos e processuais'' é voluntária e que posso retirar este consentimento a qualquer momento sem penalidades ou perda de qualquer benefício. Estou ciente também dos objetivos da pesquisa, dos procedimentos aos quais serei submetido(a), dos possíveis danos ou riscos deles provenientes e da garantia de confidencialidade e esclarecimentos sempre que desejar.
Diante do exposto expresso minha concordância de espontânea vontade em participar deste estudo.

\begin{center}
  Rio de Janeiro, \rule[-1mm]{1cm}{0.3mm} de \rule[-1mm]{3cm}{0.3mm} de \rule[-1mm]{1.5cm}{0.3mm}.\\
  \vspace{1cm}
  \rule[-1mm]{8cm}{0.3mm}\\
  (Assinatura do participante da pesquisa)\\
  \vspace{1cm}
  \rule[-1mm]{8cm}{0.3mm}\\
  (Igor de Oliveira Costa -- pesquisador responsável)
\end{center}

\newpage
\bibliography{Cap9_bibliografia.bib}

\end{document}
