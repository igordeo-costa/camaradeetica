\chapter{Objetivos}
Dada a revisão da literatura apresentada anteriormente, alguns objetivos se colocam neste trabalho, sendo o principal deles \emph{investigar o comportamento do processador humano em tempo real durante a leitura de sentenças duplamente quantificadas ambíguas ou potencialmente ambíguas}. Uma vez que a literatura apresenta hipóteses distintas sobre o comportamento do \emph{parser}, a fim de alcançar o objetivo apresentado anteriormente, este projeto tem por meta, também, prover evidência que auxilie na seleção das hipóteses em competição. Ao longo desse percurso, almeja-se angariar evidências que ajudem, ainda, em uma melhor descrição de determinadas estruturas linguísticas do português brasileiro, bem como na descrição do comportamento de \emph{todo+NP}, expressão que parece gerar grande debate, como visto na revisão da literatura.

Uma revisão mais aprofundada da literatura sobre as técnicas experimentais utilizadas no estudo de escopo de quantificadores, bem como os impactos que elas têm sobre os resultados experimentais obtidos também se enquadra nos objetivos deste trabalho. Com isso, será possível contribuir para o debate teórico e metodológico da área, uma vez que muitos dos resultados conflitantes mostrados na literatura estão vinculados ao uso de diferentes técnicas e diferentes estruturas linguísticas que não são, muitas vezes, comparáveis diretamente. Fornecer evidências que apontem para uma métrica que permita vislumbrar a comensurabilidade do fenômeno em questão também se enquadra nos objetivos deste trabalho, já que ajudam a pensar sobre como investigar o comportamento do \emph{parser}.\\
\\
As perguntas centrais que embasam a pesquisa são:

\begin{enumerate}
  \item Como o processador humano ou \emph{parser} processa sentenças com os quantificadores universal \emph{todo} e indefinido \emph{um}?
  \item O processador humano, ao processar sentenças duplamente quantificadas, atua de modo algorítimico ou se utiliza de heurísticas?
  \item O processador humano se vale de uma leitura preferencial ou \emph{default} ou mantém as leituras subespecificadas até que informação de natureza estrutural ou contextual lhe permita selecionar uma delas?
\end{enumerate}

\noindent A fim de responder tais perguntas, deve-se responder também:

\begin{enumerate}[label=(\alph*)]
  \item Quais as características estruturais e semântico-pragmáticas de sentenças com \emph{todo} e \emph{um} são relevantes a fim de se construir sentenças verdadeiramente ambíguas no que diz respeito a escopo de quantificadores?
  \item Como evitar sentenças enviesadas (semantica ou pragmanticamente), de modo que o comportamento do \emph{parser} possa ser investigado em estado natural?
  \item Quais técnicas experimentais discutidas na literatura permitiriam investigar o fenômeno em estudo de modo preciso e rigoroso?
  \item Como conciliar os resultados muitas vezes conflitantes da literatura a partir dos resultados a serem obtidos?
\end{enumerate}
