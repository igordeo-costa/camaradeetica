\chapter{Sobre o \emph{todo} no português}
O termo \emph{distributividade} pode ser definido como ``a atribuição de um predicado \textbf{a membros} de um NP [sintagma nominal, do inglês \emph{Nominal Phrase}] complexo e não ao NP complexo como um todo'' (Clifton \& Frazier, 2012)~\cite{ClifFrazier2012}. Em português, a estrutura linguística mais destacada para expressar relações de distributividade são os chamados \emph{operadores distributivos universais}, a saber: \emph{todo} e \emph{cada} (Negrão, 2002: 186-187)~\cite{neg2002}. Embora o uso de \emph{cada} seja reconhecidamente um uso distributivo, o operador \emph{todo} apresenta algumas nuances, sobretudo porque pode aparecer em pelo menos três formas distintas: (i) todo + DP plural; (ii) todo + DP singular; e (iii) todo + NP singular, conforme abaixo:

\begin{enumerate}
  \item \textbf{Todas as crianças} jogam bola.
  \item \textbf{Toda a criançada} joga bola.
  \item \textbf{Toda criança} joga bola.
\end{enumerate}

Como veremos em seguida, há indícios na literatura (Negrão, 1999 \& 2002~\cite{neg2002}; Silva, 2012~\cite{silva2012}; Rodrigues \& Marcilese, 2014~\cite{RodMarc2014}; Pires de Oliveira, 2003~\cite{PiresOliv2003}) de que esse elemento parece apresentar um comportamento errático, ora permitindo leituras coletivas ora distributivas ou mesmo genéricas.

\section{\emph{Todo} como um indefinido}
O trabalho de Negrão (2002)~\cite{neg2002} contrasta o comportamento de \emph{todo} com \emph{cada}, mostrando que, embora aparentemente apresentem comportamento semelhante, permitindo leituras distributivas, essa semelhança é apenas superficial. Isso ocorre, dentre outros motivos,  porque \emph{todo} apresenta uma leitura genérica em muitos contextos, como mostram as sentenças a seguir.

\begin{enumerate}[resume]
    \item *Cada homem é inteligente.
    \item Todo homem é inteligente.
\end{enumerate}

A última sentença não afirma que há uma série de subeventos de \emph{ser inteligente} ou que a propriedade ``inteligência'' é distribuída numa relação uma a uma com os homens. Ela diz que, em geral, genericamente, os homens são inteligentes. Essa leitura genérica também é vista quando \emph{todo+NP} não ocupa a posição de sujeito:

\begin{enumerate}[resume]
    \item *Aquele médico examinou todo paciente.
    \item *Aquele médico examinou todo paciente num horário.
 \end{enumerate}

No caso dessas sentenças, \emph{todo} não poderia subir em Forma Lógica (LF, do inglês \emph{Logical Form}) acima do VP (sintagma verbal, do inglês \emph{Verbal Phrase}) para escopar sobre o evento de \emph{examinar}. Mesmo com o acréscimo de um adjunto locativo-temporal (\emph{num horário}), que salvaria sentenças com \emph{cada} nesse contexto (vide a gramaticalidade de \emph{Aquele médico examinou cada paciente num horário}), a leitura distributiva ainda não se realiza. O único sentido possível para as sentenças acima seria aquele parafraseado por \emph{todos os pacientes} ou \emph{os pacientes como um grupo, um conjunto}.

Esse argumento se torna ainda mais forte devido ao fato de que \emph{todo+NP} não toma escopo sobre um agente da passiva, mesmo quando este vem acompanhado de um termo como \emph{diferente}, que garantiria a leitura distributiva.

 \begin{enumerate}[resume]
     \item ?Todo texto foi lido por um aluno diferente.
 \end{enumerate}

Nesse caso, segundo a autora, a sentença não recebe uma leitura distributiva, mas uma em que \emph{diferente} é lido como \emph{fora do comum}, \emph{excepcional}, ou seja, uma leitura adjetival de \emph{todo}.

Por fim, ainda há o argumento de que \emph{todo+NP} não sustenta leitura de escopo invertido quando fora do sujeito em sentenças como aquela abaixo, em que a leitura possível pode ser parafraseada por \emph{Maria tem um só vestido, o mesmo vestido, e o usa em todas as ocasiões} e não que ela teria \emph{vários vestidos, um para cada ocasião}, ou seja, uma leitura distributiva. Esse tipo de fenômeno ocorreria porque \emph{todo} não consegue subir em LF para uma posição acima de \emph{um vestido}, realizando escopo invertido.

\begin{enumerate}[resume]
    \item Maria tem um vestido para toda ocasião.
\end{enumerate}

Com isso, a autora afirma que \emph{todo+NP} só tem leitura distributiva em posição de sujeito de sentenças ativas, e que a leitura coletiva é proibida em outros contextos, já que ele não conseguiria realizar escopo invertido. Quando na posição de objeto, a leitura de \emph{todo+NP} tem de ser genérica. Ela argumenta, ainda, que essa expressão não seria, em PB, um verdadeiro quantificador, mas um indefinido.

Em trabalho posterior (Müller, Negrão \& Quadros Gomes, 2007~\cite{MullerETal2007}), a autora parece reformular a sua proposta original, argumentando que \emph{todo} é de fato um quantificador universal distributivo em todas as circunstâncias em que ocorre, seja acompanhado de NP singular (\emph{todo+NP}) seja acompanhado de DP singular ou plural (\emph{todo+DP}), não sendo ambíguo entre uma interpretação coletiva e distributiva. Na verdade, segundo elas, a leitura coletiva com \emph{todo + DP} é apenas uma ilusão. Essa ilusão adviria do fato de \emph{todo} poder quantificar sobre um nome nu (\emph{todo+NP}) ou uma descrição definida (\emph{todo+DP}). As diferentes interpretações adviriam, nessa visão, do tipo de elemento que acompanha o quantificador e não de uma propriedade inerente a ele ou ao fato de ele não ser um quantificador verdadeiro.

\section{\emph{Todo} como um quantificador modal}
Em contraste com a visão de que \emph{todo}, no português brasileiro, seria um indefinido ou que as suas diferentes leituras adviriam do fato de esse elemento poder ser acompanhado de NP ou de DP, Pires de oliveira (2003)~\cite{PiresOliv2003} defende que \emph{todo+NP} é, na verdade, um \emph{quantificador modal}, ou seja, que ele requer em seu escopo a presença de um elemento de natureza modal, comportando-se como o \emph{any} do inglês. A fim de defender esse ponto, a autora elenca algumas propriedades de \emph{todo+NP}. Abaixo revisamos a maioria dessas propriedades, visto que elas são de sobremaneira relavantes para a construção dos estímulos experimentais que utilizaremos neste estudo. Os exemplos dados, a menos que especicificado em contrário, são originais da autora.

\subsection{Flutuação}
A primeira das propriedades de \emph{todo+NP} que abordaremos é aquela chamada por Pires de Oliveira (2003)~\cite{PiresOliv2003} de \emph{flutuação}. Enquanto o quantificador universal \emph{todo+DP} pode ser trocado de posição em sentenças como as de (1), o mesmo não ocorre com \emph{todo+NP} em (2):

\begin{enumerate}
    \item a. Todas as crianças choram. \\
    b. As crianças todas choram.
    \item a. Toda criança chora. \\
    b. *Criança toda chora.
\end{enumerate}

\subsection{Retomada anafórica}
A segunda das propriedades de \emph{todo+NP} é aquela que contrasta a retomada do elemento nominal por um pronome anafórico, como nos exemplos abaixo:

\begin{enumerate}[resume]
    \item As crianças, todas elas choram.
    \item *A criança, toda ela chora.
\end{enumerate}

Isso ocorreria, segundo Pires de Oliveira (2003)~\cite{PiresOliv2003} porque \emph{todo+NP} se combinaria com um predicado, enquanto \emph{todo+DP} se combinaria com um argumento e, sendo argumento, pode se sustentar sozinho, sem a necessidade de ser ligado (\emph{bounded}) por um operador. Como a retomada anafórica só retoma indivíduos (ou seja, argumentos e não predicados), então ficaria explicado por que ela não pode ocorrer com \emph{todo+NP}. Esse mesmo motivo explicaria a diferença no caso de \emph{flotation} dado acima.

\subsection{Contextos episódicos}
Outra importante propriedade de \emph{todo+NP} está relacionada à leitura episódica ou não episódica. Observe os exemplos abaixo, dados em Pires de Oliveira (2003:367)~\cite{PiresOliv2003}:

\begin{enumerate}[resume]
    \item Toda a criançada chora. \\
    Trata de um grupo específico de crianças; \\
    Faz leitura episódica de \emph{chorar}.
    \item Toda criança chora. \\
    Trata de uma leitura genérica de \emph{chorar}; descreve uma ``lei'' sobre comportamento de crianças; \\
    Faz leitura não episódica de \emph{chorar}.
\end{enumerate}

Esse tipo de fenômeno ocorreria porque \emph{todo+DP}, por ser uma expressão definida, trata de um indivíduo particular. Esse indivíduo, no entanto, pode ser tanto \emph{atômico} quanto \emph{grupal}. Sendo \emph{atômico}, a sentença tem apenas uma leitura; sendo \emph{grupal}, a sentença é ambígua, podendo ter leitura \emph{contínua} (conta indivíduos de um conjunto) ou \emph{descontínua} (toma o indivíduo em sua totalidade), no sentido de Peres (1992). Veja a distinção nas sentenças abaixo, retiradas de Pires de Oliveira (2003: 367)~\cite{PiresOliv2003}:

\begin{enumerate}[resume]
    \item O menino todo se machucou. \\
    O mesmo que \emph{o menino inteiro se machucou} \\
    Única leitura possível: contínua.
    \item Toda a criançada chora. \\
    Pode tanto ter leitura contínua quanto a descontínua (p. 367), significando:
    \emph{O conjunto inteiro de crianças chora} ou \emph{cada criança do conjunto chora}.
\end{enumerate}

\emph{Todo+NP}, no entanto, não tem essa propriedade, nunca podendo aparecer em contexto episódico:

\begin{enumerate}[resume]
    \item *Toda a criança se machucou. \\
    \textcolor{red}{Erica, acho que a Roberta se equivocou aqui (exemplo 18, página 368... Deveria ser \emph{*Toda criança}, não?! A ideia aqui acho é a frase só poder ser lida em sentido genérico (é uma ``lei'' que \emph{toda criança se machuca}, mas não episódica, leitura descontínua, \emph{a criança se machucou}... a flexão do verbo impede a leitura genérica e a frase é ruim por isso).}
\end{enumerate}

A impossibilidade de aspecto progressivo é outra propriedade que demonstra a não ocorrência em contexto expisódico, visto que aspecto progressivo só pode ocorrer com definidos, ou seja, com \emph{todo+DP}, mas não com \emph{todo+NP}.

\begin{enumerate}[resume]
    \item Toda a criançada está brincando.
    \item *Toda criança está brincando. \\
    Essa sentença não pode ser interpretada no sentido de \emph{cada uma das crianças está brincando}, ou seja, com leitura descontínua.
\end{enumerate}

O mesmo pode ser dito sobre a impossibilidade de combinação de \emph{todo+NP} com partitivos, enquanto os sintagmas definidos podem fazê-lo explicitamente:

\begin{enumerate}[resume]
    \item Toda a criançada da festa se machucou.
    \item ?Toda criança da festa se machucou.
\end{enumerate}

\subsection{\emph{Subtrigging}}
Um outra propriedade de \emph{todo+NP} é o fato de eles poderem ocorrer na posição de objeto (ao contrário do que argumenta Negrão, 2002) desde que o nome nu esteja sendo modificado (implícita ou explicitamente), visto que \emph{todo+NP} não é sobre indivíduos particulares, mas ``sobre indívudos possíveis que podem ser caracterizados pela propriedade expressa pelo substantivo comum.'' (Pires de Oliveira, 2003: 370)~\cite{PiresOliv2003}. Por exemplo:

\begin{enumerate}[resume]
    \item Ele canta todas as canções. \\
    Trata de canções particulares: canção A, B, C, etc. Logo, é gramatical.
    \item *Ele canta toda canção. \\
    Trata da totalidade das canções possíveis. Logo, não pode ser gramatical.
    \item Ele canta toda canção que escuta no Spotify. (Exemplo meu.) \\
    Trata não do conjunto de canções possíveis, mas do conjunto de canções possíveis que foram individualmente ouvidas.
\end{enumerate}

Esse argumento é usado pela autora para descaracterizar a posição de Negrão (2002)~\cite{neg2002}, de que \emph{todo+NP} não poderia tomar escopo invertido em posição de objeto. Na verdade ele pode, desde que o nome que o acompanha esteja modificado. De acordo com essa autora, a última sentença acima não poderia ser licenciada porque haveria um conflito entre o traço [+universal] do quantificador \emph{todo} e o traço [+existencial] do predicado verbal.

Essa ideia está vinculada à proposta da autora de que tal elemento é um indefinido e não um quantificador. A possibilidade de \emph{subtrigging}, entretanto, desafia essa visão, visto que, \emph{todo+NP} não se comporta como um, como mostra o exemplo:

\begin{enumerate}[resume]
    \item Ele canta uma canção que escuta no Spotify. \\
    Essa frase só tem leitura existencial, não genérica, como é o caso da mesma sentença com \emph{toda} no lugar de \emph{uma}.
\end{enumerate}

\subsection{Licenciamento como sujeito}
Outro contraste interessante de \emph{todo+NP}, esse também apontado por Negrão (2002)~\cite{neg2002}, é o fato de esse elemento poder ser licenciado como sujeito quando o objeto é um sintagma quantificador de grupo:

\begin{enumerate}[resume]
    \item *Cada homem ama aquela mulher. \\
    Por ser integralmente distributivo \emph{cada} torna a sentença agramatical.
    \item Todo homem ama aquela mulher. \\
    Essa sentença mostra que \emph{todo+NP} não pode ser total ou somente distributivo.
\end{enumerate}

O mesmo vale para o fato, também apontado por Negrão (2002)~\cite{neg2002}, de que \emph{todo+NP} pode ser sujeito de predicados \emph{individual level}, com leitura genérica, na forma de ``lei'', enquanto o distributivo ``puro'' não pode:

\begin{enumerate}[resume]
    \item *Cada homem é inteligente.
    \item Todo homem é inteligente.
\end{enumerate}

\subsection{Modificadores}
Assim como \emph{any}, que aceita \emph{almost} e \emph{absolutely} como modificadores, \emph{todo+NP} aceita \emph{quase} e \emph{certamente} como modificadores, uma propriedade dos universais, mas não dos indefinidos. Além disso, também como \emph{any}, aceitam \emph{exceptional phares}, outra propriedade dos universais, mas não dos indefinidos.

\begin{enumerate}[resume]
    \item Quase todo menino brinca.
    \item Certamente todo menino brinca.
    \item Todo menino exceto o João brinca.
\end{enumerate}

No caso das sentenças acima, todas recebem leitura genérica, na forma de ``lei''.

\subsection{Estrutura tripartida}
Outra propriedade de \emph{todo+NP} que parece colocá-lo no grupo dos quantificadores é o fato de engendrarem estrutura tripartida.

\begin{enumerate}[resume]
    \item *João conversou com todo aluno \textbf{bravo}.
    \item João conversou com todo aluno \textbf{que estava bravo}.
\end{enumerate}

Na primeira sentença, a única leitura possível seria aquele em que \emph{todo aluno bravo possível} teve uma conversa com João. No entanto, essa leitura é impossível, visto que é uma sentença episódica (\emph{conversou}) e, como tal, exigiria uma situação particular. No caso da segunda sentença isso pode ocorrer, visto que a oração relativa restringe a uma situação particular.

\subsection{A natureza modal de \emph{todo+NP}}
Em resumo, visto que \emph{todo+NP} se comporta como um quantificador e não como um indefinido, não apresentando as propriedades deste, a autora defende que ele de fato é um quantificador, mas um quantificador excepcional, visto que requer um elemento de natureza modal em seu escopo. Um primeiro argumento nessa linha são as implicações possíveis com \emph{todo+NP} que se perdem quando ele é substituído por um indefinido:

\begin{enumerate}[resume]
    \item Ela pode cantar toda canção desse álbum. \\
    \emph{Aquarela do Brasil} é uma canção desse álbum. \\
    Logo: Ela pode cantar \emph{Aquarela do Brasil}.
    \item Ela pode cantar uma canção desse álbum. \\
    Nesse caso, a leitura genérica se perde e a implicação acima se torna falsa. \\
    Isso provaria que a qualidade quantificacional é uma propriedade do \emph{todo} e não do contexto morfológico.
\end{enumerate}

Um segundo argumento se vincula à possibilidade de \emph{todo+NP} poder tomar escopo inverso na posição de objeto, já discutido acima. Ao contrário de Negrão (2002)~\cite{neg2002}, que afirma ter a sentença abaixo apenas uma leitura, Pires de Oliveira (2003)~\cite{PiresOliv2003} argumenta que ela tem duas leituras possíveis:

\begin{enumerate}[resume]
    \item Maria tem um vestido para toda ocasião. \\
    Negrão (2002)~\cite{neg2002}: escopo invertido é impossível. \\
    Única leitura permitida é aquela em que Maria tem um único vestido e o usa em diferentes ocasiões.\\
    \\
    Pires de Oliveira (2003)~\cite{PiresOliv2003}: é possível haver uma leitura distributiva (um vestido diferente para cada ocasião).
\end{enumerate}

Dado, então, que \emph{todo+NP} seria melhor classificado como um quantificador e não como um indefinido, a autora aponta dois possíveis modos de facilitar a leitura modal, a saber: (i) o verbo portar uma pressuposição que pode ser acomodada no restritor do operador; e (ii) o NP ser modificado a fim de prover material para atuação do operador. Essa seria uma visão diferente para \emph{subtrigging}, visto que esse modificador proveria material para permitir a partição da sentença.

\section{Resultados experimentais em português}
Rodrigues \& Marcilese (2014)~\cite{RodMarc2014} buscam fazer uma revisão do tipo de resultado experimental obtido com \emph{todo+NP} (entre outros quantificadores) em contextos experimentais. O que elas reportam é que o tipo de tarefa experimental parece ter um impacto no modo como \emph{todo} é preferencialmente interpretado.

\begin{enumerate}
    \item Tarefa de seleção de imagens \\
    Leitura preferencial: coletiva \\
    Médias de 4,05 em escala de 6 ponto
    \item Tarefa de adequação sentença-imagem \\
    Leitura preferencial: Não foi possível aferir preferência \\
    86\% de leituras coletivas X 56\% de leituras distributivas
    \item Tarefa de adequação sentença-imagem evitando pareamento 1 a 1 \\
    Leitura preferencial: coletiva \\
    75\%  de leituras coletivas X 52,2\%  de leituras distributivas
    \item Tarefa de produção de imagem com interpretação livre \\
    Leitura preferencial: ??? \\
    42\% de leituras coletivas (ou genéricas?) X 47\% de distributivas X 11\% de outras
    \item  Tarefa de leitura automonitorada \\
    Leitura preferencial: coletiva \\
    Maiores tempos de reação para leitura distributiva
    \item Tarefa de Leitura automonitorada com julgamento de gramaticalidade \\
    Leitura preferencial: coletiva \\
    47,5\% de leituras coletivas X 19\% de distributivas para julgamento \emph{SIM, é sentença aceitável}
    \item Tarefa de Julgamento de gramaticalidade com resposta escalar (1-ruim a 5-bom) \\
    Leitura preferencial: Não foi possível aferir preferência \\
    Coletiva não se acumulou no topo da escala (julgamentos 4 e 5) X distributiva não se acumulou em nenhum ponto da escala
    \item Tarefa de julgamento e Tempo de Reação diante de contexto prévio \\
    Leitura preferencial: distributiva \\
    Contexto distributivo recebeu mais julgamentos \emph{SIM} \& contexto coletivo teve maiores tempos para julgamento \emph{NÃO}
    \item Tarefa de julgamento e Tempo de Reação diante de contexto prévio com ranqueamento \\
    Leitura preferencial: Não foi possível aferir preferência \\
    \emph{Todo+NP} nunca é escolhido como primeira ou última opção
\end{enumerate}

Diante de tais resultados, as autoras excluem a possibilidade de \emph{todo+NP} não permitir leituras coletivas: mesmo que não pareça haver uma leitura preferencial com esse tipo de expressão, a leitura distributiva aparece como uma possibilidade em muitos dos casos. Ademais, dependendo da tarefa experimental, até mesmo leituras coletivas podem não ser as preferidas, como no caso de maiores tempos para rejeição de figuras com leitura coletiva em experimento das próprias autoras. Além das leituras distributivas aparecerem com \emph{todo+NP}, as autoras afirmam também que vários estudos mostram outros tipos de leituras com essa expressão, como leituras adverbiais e genéricas.

Desse modo, elas afirmam que \emph{todo+NP} deve ser avaliado como um elemento de natureza indeterminada, nos moldes do que propõe Negrão (2002)~\cite{neg2002}, sendo um item subespecificado.
